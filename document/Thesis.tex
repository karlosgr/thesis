\documentclass[12pt,oneside]{uhthesis}
\usepackage{subfigure}
\usepackage[ruled,lined,linesnumbered,titlenumbered,algochapter,spanish,onelanguage]{algorithm2e}
\usepackage{amsmath}
\usepackage{amssymb}
\usepackage{amsbsy}
\usepackage{caption,booktabs}
\captionsetup{ justification = centering }
%\usepackage{mathpazo}
\usepackage{float}
\setlength{\marginparwidth}{2cm}
\usepackage{todonotes}
\usepackage{listings}
\usepackage{xcolor}
\usepackage{multicol}
\usepackage{graphicx}

\graphicspath{{Graphics/}}
\floatstyle{plaintop}
\restylefloat{table}
\addbibresource{Bibliography.bib}
% \setlength{\parskip}{\baselineskip}% 
\renewcommand{\tablename}{Tabla}
\renewcommand{\listalgorithmcfname}{Índice de Algoritmos}
%\dontprintsemicolon
\SetAlgoNoEnd

\definecolor{codegreen}{rgb}{0,0.6,0}
\definecolor{codegray}{rgb}{0.5,0.5,0.5}
\definecolor{codepurple}{rgb}{0.58,0,0.82}
\definecolor{backcolor}{rgb}{0.95,0.95,0.92}
\definecolor{keywordcolor}{rgb}{0.1,0.1,0.9}
\definecolor{stringcolor}{rgb}{0.7,0.1,0.1}

\lstdefinelanguage{json}{
    keywords={true,false,null},
    keywordstyle=\color{keywordcolor}\bfseries,
    string=[s]{"}{"}, % Las cadenas van entre comillas dobles
    stringstyle=\color{stringcolor},
    commentstyle=\color{gray},
    morecomment=[l]{//}, % Acepta comentarios tipo JS
    morecomment=[s]{/*}{*/},
    basicstyle=\ttfamily\small,
    numbers=left,
    numberstyle=\tiny\color{gray},
    backgroundcolor=\color{backcolor},
    showstringspaces=false,
    breaklines=true,
    breakatwhitespace=true,
    captionpos=b
}




\lstdefinestyle{mystyle}{
    backgroundcolor=\color{backcolour},   
    commentstyle=\color{codegreen},
    keywordstyle=\color{purple},
    numberstyle=\tiny\color{codegray},
    stringstyle=\color{codepurple},
    basicstyle=\ttfamily\footnotesize,
    breakatwhitespace=false,         
    breaklines=true,                 
    captionpos=b,                    
    keepspaces=true,                 
    numbers=left,                    
    numbersep=5pt,                  
    showspaces=false,                
    showstringspaces=false,
    showtabs=false,                  
    tabsize=4
}

\lstdefinestyle{jsonstyle}{
    backgroundcolor=\color{backcolour},   
    commentstyle=\color{codegreen},
    keywordstyle=\color{magenta}, % "true", "false", "null"
    numberstyle=\tiny\color{codegray},
    stringstyle=\color{codepurple},
    basicstyle=\ttfamily\footnotesize,
    breakatwhitespace=false,         
    breaklines=true,                 
    captionpos=b,                    
    keepspaces=true,                 
    numbers=left,                    
    numbersep=5pt,                  
    showspaces=false,                
    showstringspaces=false,
    showtabs=false,                  
    tabsize=2,
}


\lstset{style=jsonstyle}

\title{Título de la tesis}
\author{\\\vspace{0.25cm}Carlos Manuel García Rodríguez}
\advisor{\\\vspace{0.25cm}Nombre del primer tutor\\\vspace{0.2cm}Nombre del segundo tutor}
\degree{Licenciado en (Matemática o Ciencia de la Computación)}
\faculty{Facultad de Matemática y Computación}
\date{Fecha\\\vspace{0.25cm}\href{https://github.com/username/repo}{github.com/username/repo}}
\logo{Graphics/uhlogo}
\makenomenclature

\renewcommand{\vec}[1]{\boldsymbol{#1}}
\newcommand{\diff}[1]{\ensuremath{\mathrm{d}#1}}
\newcommand{\me}[1]{\mathrm{e}^{#1}}
\newcommand{\pf}{\mathfrak{p}}
\newcommand{\qf}{\mathfrak{q}}
%\newcommand{\kf}{\mathfrak{k}}
\newcommand{\kt}{\mathtt{k}}
\newcommand{\mf}{\mathfrak{m}}
\newcommand{\hf}{\mathfrak{h}}
\newcommand{\fac}{\mathrm{fac}}
\newcommand{\maxx}[1]{\max\left\{ #1 \right\} }
\newcommand{\minn}[1]{\min\left\{ #1 \right\} }
\newcommand{\lldpcf}{1.25}
\newcommand{\nnorm}[1]{\left\lvert #1 \right\rvert }
\renewcommand{\lstlistingname}{Ejemplo de código}
\renewcommand{\lstlistlistingname}{Ejemplos de código}

\begin{document}

\frontmatter
\maketitle

\include{FrontMatter/Dedication}
\begin{acknowledgements}


    Existen tantas personas a las que pienso que le debo tanto, y que contribuyeron a que yo llegara hasta aquí, que ni si quiera estoy seguro de por donde empezar.
    Quisiera empezar por mi familia, que desde que tengo memoria se han esforzado muchísimo para educarme y enseñarme de la mejor manera posible, y que sentaron las bases de lo que soy a día de hoy.
    Se podría decir que los primeros profesores de la carrera fueron ellos.
    Gracias además por todo el apoyo que me han proporcionado a lo largo de la carrera, que me ha permitido concentrarme en mi vida universitaria y mi aprendizaje sin tener muchas otras preocupaciones en la cabeza.
    Sé además que es un esfuerzo consciente de su parte, y lo valoro mucho, muchas gracias.

    Otro agradecimiento muy especial es para los profes de esta carrera, no solo a los que dedicaron al menos un poco de su tiempo a ayudarme con el desarrollo de este trabajo, ya sea guiándome a lo largo del proceso, revisando y corrigiéndome las veces que fueran necesarias o incluso algo tan simple como darme su opinión después de ojear mi trabajo.
    Sino a todos los profesores que a lo largo de la carrera, con sus clases y sus charlas, lograron motivarme e interesarme por la ciencia de la computación. Hasta el punto de que es para mí una fuente de emoción y orgullo atravesar por las asignaturas de esta carrera absorbiendo todo el conocimiento posible.
    Con la idea de poder aplicarlos en el futuro, o poder transmitirlos a otros alumnos de la misma manera que ellos lo hicieron conmigo.

    Tengo además un grupo enorme de personas especiales a las que agradecerles no solo su apoyo, sino la influencia que han tenido en mi vida y en mi persona.
    A sofi: muchas gracias por todo lo que significaste en mi vida, y por enseñarme a ser la persona que soy hoy en día, por todo el apoyo y cariño que me diste no solo durante problemas de la carrera, sino en todos los aspectos de mi vida, gracias por todas las enseñanzas, todas las risas, los chismes y los momentos compartidos.
    A rosa: gracias por todas las veces que me acogiste en tu casa y me preparaste café para que yo pudiera terminar proyectos de madrugada, gracias por todos los momentos donde me has ayudado o simplemente has estado ahí para mí.
    A todos mis amigos de la uni: gracias por ser una parte importante de todo este proceso, por todas las veces que nos reunimos a estudiar cualquiera que fuera la asignatura que teníamos prueba o proyecto esa semana, por todos los juegos de gatos bomba y los chismes compartidos, y por todas las veces que nos ayudamos durante la carrera.
    Al presi y a erne: gracias por todas las conversaciones de madrugada, donde nos pasábamos horas y horas hablando sobre cualquier tema que no tuviera que ver con la carrera.

    La verdad una de las cosas que más agradezco es que tengo tantas personas valiosas que no los puedo mencionar a todos en los agradecimientos, pero ellos saben que estan aquí.
    A mis amiguitas de diseño: fla y caro, a mis amiguitos de las candelas y del viaje a topes, a erne que es de mis amistades más viejas y a baly que es de las más nuevas.
    Y en general a todo el resto de personas que no están aquí por falta de páginas, muchas gracias por todo, porque aunque no participaron directamente en nada de este trabajo, si que aportaron mucho.





\end{acknowledgements}
\include{FrontMatter/SupervisorOpinion}
\include{FrontMatter/Abstract}
\include{FrontMatter/Contents}

\mainmatter

\chapter*{Introducción}\label{chapter:introduction}
\addcontentsline{toc}{chapter}{Introducción}
% Optimization background and history
La optimización ha sido una disciplina clave en muchos campos de la ciencia y la ingeniería a lo largo de la historia. Desde hace muchos años,
la idea de maximizar o minimizar ha sido fundamental en el cálculo y la física, no estaba formalmente consolidada. Sus primeros usos para resolver problemas prácticos
en la epóca moderna fueron durante la Segunda Guerra Mundial, donde se utilizó para
resolver asuntos de logística y planificación militar, dando origen a la investigación operativa (\textit{Operation Research}, en inglés) \cite{Petropoulos_2023}.
La investigación operativa integraba matemáticas, estadística y experimentación para optimizar operaciones militares, y fue pionera en usar métodos cuantitativos
para la toma de decisiones óptimas en organizaciones. Tomando inspiración en procesos naturales, aleatorios o en el comportamiento humano, surgieron los algoritmos metaheurísticos.
Estos algoritmos dieron lugar a nuevos métodos para resolver problemas de optimización, alcanzando soluciones aproximadas al óptimo en tiempos mucho menores \cite{Tomar2023metaheuristics}.
Este avance significó un gran salto en la capacidad de resolver problemas complejos y reales, que hasta ese momento eran intratables por los métodos exactos.


% Growing interest in optimization in recent years and machine learning influence
Dada la relevancia actual de los problemas de optimización, el interés por este campo se ha desarrollado considerablemente en los últimos años. Un estudio reveló que en los últimos 30 años la cantidad de investigaciones, árticulos y
avances importantes relacionados con este tema se ha incrementado exponencialmente \cite{weinand2021research}. El surgimiento del \textit{machine learning} como herramienta para abordar problemas complejos y de alta dimensionalidad
fue un factor clave en el campo de la optimización. La estrecha relación entre estos dos campos ha sido un factor clave en este desarrollo. Los métodos de aprendizaje han logrado modelar relaciones complejas, expandiendo
significativamente el espectro de problemas de optimización que pueden resolverse \cite{bengio2020machinelearningcombinatorialoptimization}.

% Solving optimization problems nowadays
En la actualidad existen una gran cantidad de herramientas y softwares para resolver problemas de optimización. Cada una enfocada en resolver uno o varios de los tipos de problemas
de optimización existentes (programación lineal, programación entera, programación no lineal, optimización robusta y adaptiva, etc). Estas herramientas son lo suficientemente avanzadas y eficientes para manejar problemas relativamente grandes
y complejos en tiempos razonables. En la actualidad resolver problemas de optimización complejos consta de tres pasos \cite{zhang2024solvinggeneralnaturallanguagedescriptionoptimization}. Primero un
experto en el dominio del problema debe analizar los requerimientos y objetivos prácticos, convirtiendolos en una descripción formal del problema. Después se extrae la información relevante de la descripción y se representa en alguno de los lenguajes
de modelado existentes, como \textit{Python}, \textit{AMPL}, \textit{Julia}, etc. Finalmente se usan las herramientas anterioremente mencionadas para alcanzar la solución final.

% The LLM usage in optimizations problems

Desde hace mucho tiempo ha existido un gran interés en reducir la brecha que hay entre el lenguaje natural y el modelado algebraíco o algorítmico de una computadora.
De hecho, en 1996, Eugene C. Freuder describió la idea de que “el usuario especifique el problema y el ordenador lo resuelva” como el “Santo Grial” de la programación de restricciones \cite{tsouros2023holygrail20natural}.
Hace algunos años se dieron pequeños pasos persiguiendo este objetivo, como sistemas expertos que intentaban asistir en el entendimiento e interacción de un modelo matemático usando lenguaje natural \cite{Dantzig1951}.
Sin embargo el avance en el procesamiento y entendimiento del lenguaje natural ha sido un proceso lento debido a su complejidad. Hasta hace pocos años tareas como traducción de textos, análisis sintáctico de oraciones y el
reconocimiento de entidades habían progresado bastante llegando a obtener buenos resultados. Pero todavía analizar un problema complejo, que tiene contexto, ambiguedades e ideas implícitas parecía algo lejano. A partir del año 2018
aparecieron los modelos de lenguaje preentrenados (\textit{pre-trained language models}, en inglés) como \textit{BERT} y \textit{GPT-2} que lograron un desempeño sobresalientes en tareas de procesamiento y generación de lenguaje. Estos modelos fueron escalando en tamaño
hasta llegar a los \textit{Large Language Models} (LLM). Al aumentar el número de parámetros y el tamaño del corpus de entrenamiento, estos modelos no solo escalaron en sus capacidades para entender, generar y manipular textos, sino que desarrollaron “habilidades” que los modelos
más pequeños no presentaban \cite{zhao2025surveylargelanguagemodels}. Dichas “habilidades” hacen referencia a un aumento en el desempeño al utilizar técnicas de \textit{prompting} como \textit{in-context learning} y \textit{few-shots}.


\newpage

\subsection*{Problema}

A pesar de los avances en técnicas y herramientas desarrollados recientemente en el campo de la optimización, hay un gran
problema que todavía persiste: estas herramientas requieren una modelación formal del problema para poder resolverlo. Sin embargo,
los problemas de la vida real no suelen seguir este formato, con frecuencia, describir con exactitud el objeto de optimización resulta complicado incluso para expertos.
Esta dificultad proviene, en gran medida, de la complejidad intrínseca de muchos casos prácticos. Es habitual que incluyan un elevado número de variables, restricciones
implícitas o interrelaciones complejas.
Por lo tanto, en la práctica, se necesita conocimientos especializados en el campo de la optimización y en el uso de dichas herramientas, para poder entender el problema y llevarlo
a una modelación formal.

Este proceso es generalmente costoso en cuestiones de tiempo y recursos, así como demandar la intervención de profesionales con experiencia. Debido a esto existe una barrera entre empresas, negocios o
instituciones, que no cuentan con recursos o personal experto en este campo, y la posibilidad de acceder a las tecnologías de optimización. Siendo este acceso un aspecto
significativo para facilitar la toma de decisiones, mejorar la eficiencia de sus procesos y solucionar próblematicas complejas.


\subsection*{Motivación}
Con este trabajo de tesis se realiza una propuesta para aumentar la accesibilidad a las tecnologías de optimización. Permitiendo a empresas, negocios, investigadores o cualquier persona,
solucionar problemas prácticos de su contexto de forma sencilla e intuitiva. En general, que usuarios no expertos, puedan resolver problemas de optimización a partir de una conversación.
Se busca que a través de un pequeño intercambio, se pueda extraer toda la información del problema y resolverlo automáticamente, dando en todo momento retroalimentación sobre el proceso. Esto permite
solucionar el problema incluso si el usuario no posee un entendimiento fundamental del mismo.



\subsection*{Antecedentes}
El empleo de los LLMs para permitir que usuarios no especializados aborden problemas reales complejos ha sido objeto de investigaciones previas en la Facultad de Matemáticas y Computación de la Universidad de
La Habana. En este contexto, se han desarrollado asistentes virtuales impulsados por LLMs con el propósito de facilitar la comprensión y el acceso a la información de la legislación cubana, así como de optimizar
la interacción y la extracción de datos de los complejos cuadros tabulares incluidos en los Anuarios Estadísticos de Cuba.


\subsection*{Objetivos}

El objetivo general de este trabajo es diseñar una herramienta con una estructura multiagente basados en LLMs\@. A partir de un problema de optimización descrito en lenguaje natural
la herramienta debe extraer la información relevante del problema: objetivos, restricciones, variables, etc. Usando esa información para modelar formalmente el problema y resolverlo de forma automática.
Integrado lo anterior con un sistema de retroalimentación que le permita a la herramienta rectificar errores, mejorando la correctitud de las soluciones finales. Para lograr el objetivo general se proponen los
siguientes objetivos específicos:

\begin{itemize}
    \item Realizar un análisis sobre el estado del arte enfocado en la solución de problemas de optimización a partir de lenguaje natural.
    \item Analizar las distintas tecnologías, herramientas y lenguajes de modelado utilizados para resolver los problemas de optimización actuales.
    \item El diseño y la implementación de la arquitectura multiagente propuesta.
    \item El diseño de una interfaz visual para la interacción del cliente.
    \item La implementación de un sistema de manejo y prevención de errores.
    \item Evaluar la efectividad de la herramienta propuesta, y su rendimiento comparado a otras soluciones.
\end{itemize}


\subsection*{Estructura de la tesis}
El resto del trabajo de tesis está estructurado de la siguiente manera. En el Capítulo~\ref{chapter:state-of-the-art} se realiza una comparación de los diferentes enfoques utilizados para resolver problemas de optimización
a partir de lenguaje natural. Se analizan las principales técnicas de cada enfoque para cada objetivo específico del problema y se muestran las principales dificultades existentes en este campo. Luego, en el Capítulo~\ref{chapter:proposal}
se muestra la arquitectura propuesta, incluyendo una análisis de las estrategias y técnicas utilizadas. En el Capítulo ~\ref{chapter:implementation} se exponen los resultados alcanzados por el modelo con respecto a otros trabajos similares, destacando
como se afecta el desempeño a partir de las las diferentes técnicas utilizadas. Por último se presentan las conclusiones y recomendaciones futuras.
\chapter{Estado del Arte}\label{chapter:state-of-the-art}

\section{Modelado de Optimización}
El análisis de la literatura revela que la creación de modelos de optimización a partir de lenguaje natural se centra en dos desafíos conceptuales principales.
El primero es la Extracción Semántica y de Características, que implica la identificación precisa de la información y las restricciones a partir de la descripción textual.
El segundo es la Selección del Paradigma de Modelación, que se refiere a la elección de la estructura y el tipo de representación formal (matemática) del problema.
Ambos pilares serán abordados detalladamente en esta sección, utilizando los marcos teóricos de la investigación relacionada.


\subsection{Enfoques de Modelado}
La competencia \textit{NL4Opt} (\cite{ramamonjison2023nl4optcompetitionformulatingoptimization}) fue creada con el objetivo de explorar diferentes técnicas para modelar matemáticamente un problema de optimización lineal a partir de una descripción en lenguaje natural.
Para realizar este proceso, en esta competencia se propone una metodología basada en dos sub-tareas principales. La primera de ellas es \textit{Named Entity Recognition} (NER), la cual consiste en etiquetar las entidades fundamentales en la descripción del problema: restricciones, variables, objetivos, etc.
La segunda sub-tarea es modelar el problema a partir del texto etiquetado, usando una representación intermedia (IR) basada en etiquetas \textit{XML}.
Para la realización de ambas tareas se propone el uso de modelos como \textit{BERT} (\cite{devlin2019bert}) y \textit{XLM-R} (\cite{conneau2020xlmr}).



En los trabajos presentados por~\cite{ning2023novelapproachautoformulationoptimization} y~\cite{highlightingnamedentities} se sigue la metodología propuesta en \textit{NL4Opt} para la modelación a partir de lenguaje natural, aunque el segundo artículo solo se enfoca en la segunda sub-tarea.
En el trabajo de~\cite{ning2023novelapproachautoformulationoptimization} se utiliza XLM-R como modelo base para el reconocimiento y etiquetado de entidades. A este se le aplica además ajuste fino (\textit{fine-tuning}) adoptando un entrenamiento adversarial para mejorar la capacidad de generalización del modelo.
Los autores también presentan una serie de reglas para realizar un post-procesamiento del texto etiquetado.
Para la segunda sub-tarea ambos trabajos emplean diferentes versiones de \textit{BERT} para transformar el texto etiquetado en la representación intermedia en un solo paso.
Ambos trabajos lograron buenos resultados al evaluarse de forma \textit{end to end} en el conjunto de problemas presentados en \textit{NL4Opt}.
Sin embargo, estos resultados contrastan con los publicados por~\cite{ramamonjison2023nl4optcompetitionformulatingoptimization} quienes realizan una comparación de la eficacia entre las mejores soluciones siguiendo la metodología presentada en la competencia y las capacidades de un gran modelo de languaje como \textit{ChatGPT} para resolver la misma tarea.
Se utilizó \textit{ChatGPT-3.5-turbo} y se llegó a la conclusión de que este modelo superó a dichas soluciones sin necesidad de \textit{fine-tuning} ni un etiquetado previo de entidades.



%% Prompting

Analizando la literatua estudiada posterior a los trabajos antes mencionados, es posible deducir que la metodología propuesta en \textit{NL4Opt} resultó mucho menos eficaz que otros enfoques basados en grandes modelos de lenguaje.
Es posible que tanto las conclusiones presentadas por \cite{ramamonjison2023nl4optcompetitionformulatingoptimization}, así como la acelerada evolución de las capacidades de los LLM en los últimos años (\cite{chen2025surveyscalinglargelanguage}) hayan contribuido a descontinuar la metodología presentada en \textit{NL4Opt} y a utilizar grandes modelos de lenguaje.

En general la mayoría de los estudios y artículos consultados optan por utilizar LLMs para la modelación a partir de lenguaje natural.
Sin embargo, las estrategias empleadas para inducir al LLM a realizar esta tarea varían considerablemente.
A continuación se abordan los enfoques principales seguidos en la literatura consultada.


Uno de estos enfoques es utilizar técnicas de \textit{prompt engineering} para que el LLM genere el modelo directamente a partir de la descripción del problema.
Algunos trabajos simplemente utilizan instrucciones directas y sencillas, para instruir al modelo a generar código en lenguajes de modelado, como \textit{MiniZinc} o \textit{CPMpy}, a partir de la descripción en lenguaje natural (\cite{almonacid2023automaticoptimisationmodelgenerator}, \cite{tsouros2023holygrail20natural}).
Otros utilizan un enfoque por etapas, donde primero el LLM clasifica las variables y restricciones, y luego se genera una modelación matemática (\cite{li2023synthesizingmixedintegerlinearprogramming}).

En trabajos como \textit{OptiMUS} presentados por \cite{ahmaditeshnizi2025optimus03usinglargelanguage} se utilizan y exploran algunas ideas interesantes:
\begin{itemize}
    \item Se instruye al LLM a extraer la información relevante por partes: se extraen primero los parámetros y variables y luego las restricciones y objetivo.
    \item No se limitan a identificar y extraer variables, también se le encomienda al modelo identificar dominios, tipos e información contextual.
    \item Se incluyen en el \textit{prompt} un conjunto de ejemplos de salidas esperadas, facilitando al LLM devolver la estructura requerida de la modelación.
\end{itemize}

Otra característica que presenta \textit{OptiMUS} es el uso de \textit{reflective prompting} (\cite{shinn2023reflexionlanguageagentsverbal}) para inducir al modelo a analizar su salida reflexivamente.
Facilitando al LLMs identificar y corregir sus propios errores en la modelación del problema. Se concluye que el uso de \textit{reflection} reduce significativamente los errores de modelación.

En \textit{OptiMUS} además se experimenta con el uso de \textit{Retrieval Augmented Generation} (RAG) (\cite{lewis2021retrievalaugmentedgenerationknowledgeintensivenlp}) con el objetivo de enriquecer el \textit{prompt} con información y ejemplos relevante o similares a la tarea a realizar, que se buscan y extraen de una base de datos preparada.
Prueban el uso de RAG en dos fases de la modelación: la extracción de las cláusulas (restricciones y objetivos), y durante la modelación de las mismas.
De esta manera evaluan la influencia de esta técnica en la capacidad de modelar correctamente las cláusulas del problema, al encontrar modelaciones similares e incluirlas en el \textit{prompt}.
Sin embargo, en los resultados obtenidos con el uso de RAG se muestran inconsistencias en la eficiencia del LLM para modelar el problema, por tanto se desestimó su uso en la solución final de \textit{OptiMUS}.


Por su parte, \cite{xiao2024chainofexperts} en su trabajo \textit{Chain-of-Experts} (CoE) proponen una solución basada en una estructura cooperativa multi-agente.
Este enfoque descompone la tarea de modelado en responsabilidades granulares, las cuales se asignan a un conjunto de agentes con conocimientos de dominio específicos, llamados “expertos”.
Similar a \textit{OptiMUS}, en CoE se extrae la información relevante del problema por partes, para ello se designan varios expertos a esta tarea, cada uno con el objetivo de extraer una característica esencial del problema (variables, parámetros, restricciones, etc).
En este trabajo además se utilizan técnicas de \textit{prompting} como \textit{chain of thought} (CoT) (\cite{wei2023chainofthoughtpromptingelicitsreasoning}) para potenciar las capacidades de razonamiento de los “expertos”, mejorando así sus desempeños en las tareas de extracción y modelación.
Este estudio aporta una comparativa entre la eficacia del enfoque multiagente propuesto frente a una estrategia basada en emplear diferentes técnicas de \textit{prompting} como \textit{chain of thought}, \textit{tree of thought}, \textit{reflection} en un único modelo de lenguaje.

Es importante destacar que todos los trabajos antes mencionados solo utilizan como modelos base a ChatGPT-3.5-turbo, ChatGPT-4 o ChatGPT-4o. Ya que fueron publicados antes de la llegada de los llamados “modelos razonadores” como la serie \textit{o} de \textit{OpenAI} o los últimos modelos de \textit{Gemini}.


El enfoque alternativo a usar técnicas de \textit{prompting}, es aplicar \textit{fine-tuning} a grandes modelos de lenguaje de código abierto, con el próposito de especializarlos en transformar lenguaje natural en modelos de optimización.
En el trabajo propuesto por \cite{huang2025orlmcustomizableframeworktraining} se presenta \textit{ORLM}, una herramienta que modela problemas de optimización usando modelos de código abierto de tamaño reducido, entrenados con una gran cantidad de datos.
Se diseña y presenta un método llamado \textit{OR-instruct}, con el objetivo de generar una gran cantidad de datos necesarios para el entrenamiento de los modelos. Se parte de un número reducido de problemas y se emplean diferentes técnicas de \textit{data augmentation} y validación para generar nuevos problemas, logrando generar 30000 problemas de optimización para el entrenamiento de los modelos.
Se experimenta con el uso de diferentes modelos como: Mistral-7B, Deepseek-Math-7B-Base y LLaMA-3-8B AI, donde todos alcanzan un rendimiento similar.
En este estudio también se compara la efectividad de esta propuesta con respecto a otras soluciones mencionadas anteriormente. Se concluyó que \textit{ORLM} supera por un amplio margen a técnicas de \textit{prompting} aplicadas sobre un LLM, y además logra una ligera mejora sobre el desempeño de trabajos como \textit{Chain-of-Experts} y \textit{OptiMUS}.


\cite{jiang2025llmopt} emplean esta misma estrategia para modelar problemas de optimización. Se opta por usar Qwen1.5-14B (\cite{bai2023qwentechnicalreport}), un modelo ligeramente más grande a los empleados en \textit{ORLM}.
Al igual que en \textit{ORLM} se realiza una comparación con los métodos basados en \textit{prompts}, llegando a una misma conclusión. Los resultados de este trabajo en distintos conjuntos de datos de problemas de optimización superan a los alcanzados por los trabajos que modelan aplicando \textit{prompt engineering} a los modelos de lenguaje.
Otro aporte interesante de este trabajo es el análisis de la efectividad de ChatGPT-o1, un modelo “razonador”, en esta tarea. Aunque no se logra realizar un gran número de experimentos y comparaciones debido a la baja accesibilidad de la \textit{API} del modelo en ese momento, si se propone realizar un estudio de las capacidades de estos modelos “razonadores” en un futuro.

A pesar de que aplicar \textit{fine-tuning} sobre modelos de lenguaje de código abierto, para traducir descripciones en lenguaje natural a modelos de optimización, ha demostrado un rendimiento sobresaliente, esta aproximación presenta varias desventajas significativas que deben tenerse en cuenta.
Una de ellas es la escasez y el alto costo de los datos de entrenamiento. A diferencia de tareas lingüísticas con abundantes corpus públicos, los problemas de optimización completos (enunciado, variables, restricciones y soluciones de referencia) suelen existir en repositorios dispersos,
o dentro de entornos industriales confidenciales. Esta carencia de datos introduce riesgos de sobreajuste a dominios limitados (p. ej., solo logística o sólo programación lineal) y reduce la capacidad de generalización del
modelo. Además, el proceso de \textit{fine-tuning} sobre millones de parámetros exige recursos computacionales considerables que incrementan significativamente el costo monetario.

\subsection{Modelación Formal}
En los trabajos consultados se exponen diversas formas de modelar formalmente, o semi-formalmente un problema de optimización, sin embargo en muchos de ellos no se aborda con demasiado detalle que estructura siguen estas modelaciones o que características presentan.
A continuación se presenta una caracterización de las diferentes opciones de modelación encontradas en la literatura estudiada:

\begin{itemize}
    \item Modelación usando representaciones intermedias (IR) (\cite{ramamonjison2023nl4optcompetitionformulatingoptimization}, \cite{ning2023novelapproachautoformulationoptimization}, \cite{highlightingnamedentities}, \cite{ramamonjison2022augmentingoperationsresearchautoformulation}), esta representación tiene la ventaja de ser \textit{parseable}, faciltando la validación del modelo y su transformación a otras representaciones. Sin embargo, como se mencionó anteriormente, trabajos recientes no usan esta representación.
    \item Representación usando lenguajes de modelado como MiniZinc, MindOpt o CPMpy (\cite{tsouros2023holygrail20natural}, \cite{almonacid2023automaticoptimisationmodelgenerator}).
    \item Modelación matemática (\cite{ahmaditeshnizi2025optimus03usinglargelanguage}, \cite{bertsimas2024robustadaptiveoptimizationlarge}, \cite{huang2025orlmcustomizableframeworktraining}, \cite{zhang2024solvinggeneralnaturallanguagedescriptionoptimization}, \cite{li2023synthesizingmixedintegerlinearprogramming}, \cite{xiao2024chainofexperts}), esta representación formal es la más usada dentro de los trabajos consultados, aunque su estructura varía en dependecia de la metodología usada. En algunos casos se usa simplemente una plantilla con una estructura sencilla (Variables, Restricciones, Objetivo) para que el LLM rellene, en otros casos se estructura de forma un poco mas detallada utilizando  código \textit{MarkDown} o \textit{Latex} para representar las cláusulas (\cite{jiang2025llmopt}, \cite{ahmaditeshnizi2025optimus03usinglargelanguage}).
    \item Modelación usando formatos específicos como \textit{JSON} (\cite{thind2025optimaioptimizationnaturallanguage}, \cite{ahmaditeshnizi2023optimusoptimizationmodelingusing}), el uso formatos como \textit{JSON} permite estructurar la modelación de manera formal y organizada. De esta forma permite representar no solo la modelación matemática, sino también el contexto y características de cada cláusula y variable, así como los metadatos del problema.
\end{itemize}

Generalmente, cualquier problema de optimización se puede modelar matemáticamente usando variables, restricciones y objetivos. Sin embargo, en el trabajo de \cite{jiang2025llmopt} se presenta una formulación de cinco elementos: conjuntos, parámetros, variables, restricciones y objetivo.
En este estudio se afirma que el uso de esta re-presentación resulta en una formulación mas precisa del problema, favoreciendo la legibilidad y comprensión del modelo. Se demuestra además, que la formulación de cinco elementos es más facil de entender y generar por un LLM que la modelación matemática.
Esto se concluye al comparar el rendimiento de un modelo de lenguaje para modelar un conjunto de problemas de optimización alternando entre estas dos variantes.

\section{Enfoques de Solución}
Habiendo abordado en la sección anterior que técnicas y estrategias se usan para modelar un problema de optimización a partir de lenguaje natural, se pretende exponer en esta sección cómo encontrar una solución al problema de forma directa.
Se exploran dos caminos de solución fundamentales: por un lado, el uso de LLMs para generar el código y el modelo formal que herramientas externas o solvers se encargan de optimizar; y por otro, el uso del LLM como el propio motor de optimización, capaz de inferir la solución sin necesidad de intermediarios.

\subsection{Integración Modelos - Herramientas Externas}
En la literatura consultada, el enfoque más utlizado para resolver directamente un problema de optimización, a partir de su descripción en lenguaje natural fue una estrategia que consta de dos pasos.
Primero se modela el problema usando diferentes técnicas abordadas en la sección anterior, luego a partir de esa modelación formal se genera y ejecuta código específico, empleando herramientas o solucionadores (\textit{solvers}) externos.
Esta estrategia puede ilustrarse de forma general como se muestra en la Figura \ref{fig: pipeline}


\begin{figure}[h]
    \centering
    \makebox[\textwidth]{
        \includegraphics[width=1.2 \textwidth]{pipeline.png}
    }
    \caption{Estrategia general de solución}
    \label{fig: pipeline}
\end{figure}

En base a que las estrategias de modelado ya se abordaron en la sección anterior, esta subsección se centrará en analizar los enfoques empleados para resolver un problema de optimización a partir de una modelación formal.

Una estrategia fundamental, identificada en diversos trabajos de investigación, es el proceso de depuración o \textit{debugging} (\cite{ahmaditeshnizi2025optimus03usinglargelanguage}, \cite{zhang2024solvinggeneralnaturallanguagedescriptionoptimization}, \cite{jiang2025llmopt}, \cite{huang2025orlmcustomizableframeworktraining}, \cite{thind2025optimaioptimizationnaturallanguage}).
Una vez que los \textit{solvers} y herramientas externas ejecutan el código de modelado generado, es posible que estas lancen errores, debido a sintaxis incorrecta en el código o problemas ocurridos en tiempo de ejecución.
Estos errores se capturan y se utilizan como retroalimentación, permitiendo refinar y corregir el código generado inicialmente.
Esta estrategia ayuda a corregir alucionaciones producidas por los modelos de lenguaje en la generación de código, así como manejar situaciones inesperadas en la ejecución.

Otra técnica bastante empleada es el análisis cualitativo de los resultados. Dada la solución hallada por el \textit{solver} se le realiza una consulta a un modelo de lenguaje, donde se le pide analizar si los resultados obtenidos tienen sentido dado el problema inicial.
Esto permite capturar errores en la solución que no dependen de un código incorrecto, sino de un modelo mal traducido a código o mal generado. A partir de la retroalimentación del modelo es posible intentar corregir el problema, independientemente si es un problema del paso de modelación o del paso de ejecución.

Trabajos como \textit{OptimAI} (\cite{thind2025optimaioptimizationnaturallanguage}) incluyen además una etapa de planificación entre los pasos de modelación y generación-ejecución de código.
En esta etapa se realiza un análisis de las características del problema y los \textit{solvers} disponibles con el objetivo de maquetar distintas vias de solución, además de permitir escoger un \textit{solver} adecuado para el problema.
Además implementan un mecanismo de decisión que permite cambiar entre estrategias teniendo en cuenta cuantas iteraciones de depuración se han realizado.
Este trabajo, a diferencia de otros que solo se enfocan en resolver problemas de programación lineal (LP) o programación entera mixta (MILP), puede manejar un amplio rango de problemas combinando diferentes \textit{solver} y herramientas.

\subsection{LLMs como Optimizadores}
En el trabajo de~\cite{yang2024largelanguagemodelsoptimizers} se introduce \textit{Optimization by PROmpting} (OPRO), un enfoque que emplea los grandes modelos de lenguaje
como optimizadores. Esta metodología aprovecha tanto la capacidad de los LLM para procesar lenguaje natural como su habilidad para el aprendizaje en contexto, especialmente, para inferir
patrones a partir de ejemplos (\cite{mirchandani2023largelanguagemodelsgeneral}).
El objetivo es lograr que el modelo de lenguaje genere directamente candidatos a soluciones de un problema de optimización.
Para llevar a cabo este propósito se diseña un proceso iterativo, donde en cada ciclo el LLM recibe como entrada un \textit{meta-prompt} con dos informaciones claves:
la descripción del problema en lenguaje natural, y un conjunto de soluciones encontradas anteriormente.
Este conjunto de soluciones poseen un puntaje asociado, y están ordenadas de forma ascendente, facilitando al modelo de lenguaje seguir la trayectoria de optimización de las soluciones.
Una vez que el LLM propone una serie de soluciones nuevas, un módulo evaluador califica esas soluciones y las añade al \textit{meta-prompt} para la siguiente iteración.

La eficacia del método OPRO fue evaluada en dos problemas de optimización clásicos: la regresión lineal y el problema del viajante, o como se conoce en inglés: \textit{the traveling salesman problem} (TSP).
En instancias de pequeña escala, el enfoque obtiene resultados prometedores, demostrando así la capacidad de los LLMs para abordar funciones objetivo diversas utilizando únicamente prompts.
No obstante, el estudio también identifica limitaciones significativas. Entre ellas destacan la dificultad para escalar a problemas de mayor tamaño debido al límite de la ventana de contexto, y una menor eficacia al tratar con funciones objetivo complejas o grandes cantidades de variables,
lo que puede conducir al estancamiento en la búsqueda de soluciones de alta calidad.
Finalmente, se concluye que este enfoque aún no supera en rendimiento a los algoritmos y \textit{solvers} especializados existentes.

Otros trabajos como los publicados por~\cite{huang2024exploringtruepotentialevaluating} y \cite{guo2024optimizinglargelanguagemodels} también evaluan a los grandes modelos de lenguaje como optimizadores, aunque presentan pequeños cambios en su metodología con respecto a OPRO. Entre las modificaciones podemos mencionar la exploración de las capacidades del modelo de encontrar soluciones nuevas siguiendo estrategias como \textit{Grid Search}, \textit{Gradient Descent} o \textit{Hill Climbing}.
Ambos trabajos llegan a conclusiones similares a las alcanzadas por \cite{yang2024largelanguagemodelsoptimizers}, destacando el potencial de los LLMs para resolver eficientemente instancias pequeñas de problemas conocidos de optimización.
Sin embargo también plantean que su eficacia decrece ampliamente al escalar el tamaño y complejidad de los problemas, dificultando su aplicación en la práctica.

\section{Análisis General}
El estudio de la literatura evidencia un consenso claro en la utilidad de los Grandes Modelos de Lenguaje sobre otros métodos para la modelación de problemas de optimización.
Se identifican dos enfoques principales para entrenar a un LLM en la tarea de modelación, con un balance importante: el \textit{fine-tuning} ofrece mejor rendimiento, pero a expensas de un alto costo computacional y la dependencia de datos escasos; mientras que \textit{in-context learning} es más viable y generalizable, logrando alta eficacia mediante arquitecturas de agentes complejas y técnicas avanzadas como la reflexión.
Paralelamente, la estrategia de solución más robusta se centra en una arquitectura dual: el modelo de lenguaje se encarga del modelado, y solvers externos especializados ejecutan el código, utilizando mecanismos de retroalimentación y depuración para corregir los errores en tiempo de ejecución.
También es importante tener en cuenta que no se ha explorado completamente el potencial de los nuevos modelos “razonado-res” en esta tarea de solución y modelación de problemas de optimización, lo cual representa un tema relevante en el que indagar.



\chapter{Propuesta}\label{chapter:proposal}
En este trabajo se presenta una propuesta de aplicación para resolver problemas de optimización a partir de una interfaz lingüística.
Se emplea una arquitectura multi-agente como las que se mostraron en el capítulo anterior para procesar el lenguaje natural, y a partir de este modelar el problema.
Esta arquitectura utiliza un sistema centralizado, donde existe un agente central que delega tareas específicas a otros agentes, como se muestra en la figura \ref{fig: architecture}.


\begin{figure}[h]
    \centering
    \includegraphics[width=0.8\textwidth]{architecture.png}
    \caption{Arquitectura propuesta}
    \label{fig: architecture}

\end{figure}

El diseño incluye varios métodos para validar y manejar errores, con el objetivo de mitigar las alucinaciones cometidos por los modelos de lenguaje. Entre ellos podemos mencionar la retroalimentación usando los errores en tiempo de ejecución, para corregir el código generado, o el análisis reflexivo de la modelación del problema.
A partir de las opciones de modelación que se exponen en el estado del arte, se generó una estructura de modelado formal pero que permite mantener un vínculo de cada cláusula con su contexto, así como almacenar metadatos del problema. Otra de las carácteristicas que posee esta propuesta es la posibilidad de manejar distintos tipos de problemas de optimización intregando distintas herramientas y solucionadores en el entorno.
Esto permite tomar diferentes cursos de acciones en base a si es un problema de programación lineal, programación entera o satisfacción de restricciones. Un detalle a tener en cuenta es que se tiene en cuenta un tiempo límite establecido por el usuario para resolver el problema. Este límite también modifica la estrategia de solución del problema, influyendo en si se utilzan enfoques no exactos pero más rápidos para resolver el problema, en caso de que el margen de tiempo sea pequeño. O en caso de que no sea un problema, utilizar métodos exactos que garantizen la mejor solución.


\subsection*{Agente Conversacional}
La primera capa de la solución propuesta es un agente que se encarga de hacer de interfaz lingüística con el usuario. Con este agente se intenta suplir la falta de conocimiento o experticia de los usuarios, al intentar solucionar problemas prácticos.
La principal tarea de este agente es analizar el problema planteado por el usuario, y valorar potencial información faltante. Si considera que podría no haberse proporcionado algún dato escencial referente al problema, se encarga de formular una serie de preguntas
para que el usuario conteste. Esto permite que el usuario pueda proveer la información necesaria para resolver el problema, sin necesidad de conocer cúal información es necesaria o no.
Una vez recopilada toda la información pertinente, el agente la consolida en una descripción estructurada en lenguaje natural. A continuación, esta descripción es sometida a un proceso de refinamiento, donde se instruye al modelo, mediante el uso de ejemplos (few-shot prompting),
para que transforme el texto e incremente su nivel de explicitud. Este proceso incluye la estandarización de la terminología, la inserción de anotaciones sobre la naturaleza de las variables y la conversión de expresiones coloquiales a su equivalente lógico-matemático (p. ej., transformar
“x no puede superar y” en “x es menor o igual que y”). Como parte de esta investigación, se evaluará la influencia de esta transformación en la eficacia del sistema para modelar correctamente los problemas, buscando así validar la hipótesis de que un mayor grado de explicitud en la formulación
del problema potencia significativamente la capacidad de razonamiento de los Grandes Modelos de Lenguaje.

\subsection*{Modelación}
Para modelar el problema de optimización se optó por una estructura específica, utilizar un formato \textit{JSON}, presentada como \textbf{FNP}.
Se utilizó como base \textit{Structured Natural Language Optimization}(\textit{SNOP}) \cite{ahmaditeshnizi2023optimusoptimizationmodelingusing}, un formato usado para almacenar problemas de optimización de manera estructurada.
Esta estructura permite almacenar metadatos del problema, así como mantener una relación entre la formalidad del problema y su contexto.
Al utilizar un formato \textit{JSON}, se facilita el manejo de esta modelación por lenguajes de programación así como por los modelos de lenguaje.
A continuación se muestran los campos y detalles de la estructura propuesta:

\begin{itemize}
    \item \textbf{Tipo de problema:} Califica el problema según en LP (programación lineal), MILP (programación lineal entera mixta), NLP (programación no lineal), MINLP (programación no lineal entera mixta), CS (satisfacción de restricciones).
    \item \textbf{Tipo específico:} Este campos se utiliza para clasificar problemas más complejos y específicos como planificación, enrutamiento, etc. Para el resto de problemas este campo estará vacío.
    \item \textbf{Tiempo:} El tiempo límite para resolver el problema especificado por el usuario.
    \item \textbf{Conjuntos:} Aquí se almacenan conjuntos de elementos, sobre los cuales es posible iterar en el problema, por ejemplo: las diferentes ciudades de un país, los diferentes tipos de un producto, etc.
    \item \textbf{Parámetros:} Los parámetros de un problema son datos estáticos relevantes para la modelación de este, pueden ser datos únicos o datos que dependan de elementos de un conjunto.
    \item \textbf{Variables:} En este campo se modelan todo el conjunto de variables del problema. De cada variable se almacena el tipo(entera o continua), el dominio de la variable. También es posible modelar una variable por cada elemento de un conjunto de forma iterativao escalarlo creando una variable por cada par de elementos de conjuntos distintos, o aún mas. Esto es posible usando un campo llamado \textit{index} dentro de la variable, que denota el o los conjuntos sobre los que se deben iterar.
    \item \textbf{Restricciones:} De cada restricción se almacenan su tipo(linear o no linear) y su formulación matemática, que referencia a las variables y parámetros antes declarados.
\end{itemize}


En cada declaración de cláusulas o entidades en esta estructura se mantiene una descripción de la variable en lenguaje natural, con el objetivo de mantener el contexto del problema en la modelación. Un ejemplo completo de esta estructura de modelado puede encontrarse en el anexo \ref{}


\subsection*{Detalles de implementación}


\subsubsection{Interfaz gráfica}
Como parte de la solución se desarrolló una interfaz gráfica web utilizando el framework \textit{Next.js}. Esta interfaz permite al usuario interactuar con la aplicación a través de una conversación, ingresando los detalles del problema y recibiendo retroalimentación.



\subsubsection{Modelos de lenguaje}
En el plano de la implementación, se emplearon los Grandes Modelos de Lenguaje (LLM) \textit{Gemini 2.0 Flash} y \textit{Gemini 2.5 Pro}. La selección de estos modelos, desarrollados por Google, se fundamentó en un criterio pragmático: la disponibilidad de una Interfaz de Programación de Aplicaciones (API) con un nivel de acceso gratuito, lo cual fue un factor determinante para la viabilidad del proyecto.
Estos LLM constituyen el motor computacional de los agentes de modelación y de generación de código. El criterio para emplear un modelo u otro se establece en función de la complejidad de la tarea: el modelo más ágil y económico se asigna a las operaciones de menor dificultad, mientras que el de mayor capacidad se reserva para aquellos procesos que demandan un razonamiento más avanzado.

\subsubsection{Julia}
Como lenguaje de programación para modelar y resolver el problema se escogió Julia. Un lenguaje de programación de alto rendimiento diseñado para la computación técnica y científica.
Se escogió este lenguaje por las siguientes ventajas que ofrece:
\begin{itemize}
    \item Posee un paquete de modelado de optimización llamado \textit{JuMP}, con una sintaxis declarativa que permiten definir variables y restricciones muy similar a una modelación teórica.
    \item Integración sencilla con numerosas herramientas de optimización y soluciona-dores (Ipopt, NLopt, Cbc, HiGHS, SCIP, etc) a través de un paquete llamado \textit{MathOptInterface}.
    \item Compilación Just-In-Time (JIT) sobre LLVM, que proporciona velocidad comparable a C/Fortran en fases de solución y en evaluaciones repetidas de modelos grandes.
    \item Diferenciación automática integrada (ForwardDiff, ReverseDiff), para obtener gradientes y Hessianas exactas en problemas no lineales sin código manual adicional.
\end{itemize}
Dada la amplia gama de problemas que la solución propuesta es capaz de abordar, se ha implementado una arquitectura modular basada en cuatro entornos de ejecución de Julia independientes. El propósito de esta segmentación es aislar las bibliotecas y dependencias específicas para cada categoría de problema, una práctica recomendada para prevenir conflictos e incompatibilidades entre paquetes de software.
Estos entornos se han especializado de la siguiente manera:
\begin{itemize}
    \item Programación Lineal y Entera Mixta: Un entorno dedicado a problemas de Programación Lineal (LP) y Lineal Entera Mixta (MILP).
    \item Programación no Lineal y Entera Mixta: Un segundo entorno para problemas de Programación no Lineal (NLP) y no Lineal Entera Mixta (MINLP).
    \item Satisfacción de Restricciones: Un tercer entorno especializado en Problemas de Satisfacción de Restricciones (CS, por sus siglas en inglés, Constraint Satisfaction).
    \item Metaheurísticas: Finalmente, un cuarto entorno que agrupa un conjunto de herramientas basadas en metaheurísticas. Este último está diseñado para la resolución aproximada y eficiente de problemas de alta complejidad, donde se prioriza la rapidez en la obtención de una solución de alta calidad por sobre la garantía de optimalidad.
\end{itemize}

\subsection*{Agente Supervisor}
El Agente Supervisor constituye el núcleo de la arquitectura de la solución propuesta, siendo responsable de la orquestación y ejecución del flujo de trabajo completo. Su función principal es coordinar el proceso, delegando tareas específicas a los agentes subordinados: el Agente de Modelación \textit{fnp} y el Agente de Generación de Código.
Una vez que el problema ha sido modelado, el Agente Supervisor procede a validar y analizar dicho modelo para identificar sus características clave, tales como el tipo, tamaño y complejidad del problema de optimización. Con base en este análisis, selecciona el conjunto de herramientas y estrategias más adecuadas, decidiendo sobre el solucionador (solver) compatible, el entorno de ejecución óptimo y otros componentes necesarios para la resolución.
Tras esta fase de planificación, el Agente Supervisor instruye al Agente de Generación de Código para que produzca el código fuente en Julia. La elección de este lenguaje presenta un desafío inherente: al ser relativamente menos popular, los Grandes Modelos de Lenguaje (LLM) tienden a mostrar un rendimiento inferior en la generación de su código, debido a la escasez de datos de entrenamiento disponibles, tal como se menciona en \cite{cassano2024knowledgetransferhighresourcelowresource}. Para mitigar esta desventaja, durante la fase de generación de código se le provee al LLM una serie de ejemplos cuidadosamente diseñados. Se ha desarrollado un conjunto de prompts específicos, cada uno enfocado en demostrar y explicar las diferentes sintaxis para programar componentes del modelo de optimización, como variables, restricciones y funciones objetivo.
Posteriormente, el Agente Supervisor ejecuta el código generado en el entorno previamente configurado y captura la salida. Esta salida puede corresponder a los resultados del problema o, en su defecto, a un error arrojado durante la ejecución. En caso de error, se inicia un ciclo iterativo de depuración: el Agente Supervisor retroalimenta al Agente de Generación de Código con el mensaje de error, solicitándole que corrija el código para solventar el fallo. Este proceso se repite hasta que el programa se ejecute con éxito o se alcance un número máximo predefinido de intentos.
El propósito fundamental de este mecanismo de revisión y corrección automatizada es reducir la incidencia de errores derivados de las "alucinaciones" del modelo de lenguaje. Esta estrategia resulta particularmente crucial en el contexto de Julia, un lenguaje donde, como se ha mencionado, el riesgo de que un LLM genere código incorrecto o inexistente es considerablemente mayor.
\chapter{Detalles de Implementación}\label{chapter:implementation}


\section{Infraestructura}
Para la solución propuesta se implementó un servidor usando \textit{FastApi}, un framework que permite crear aplicaciones web de forma sencilla y rápida usando \textit{Python}.
Este servidor se montó en un contenedor de \textit{Docker} junto a los entornos y paquetes necesarios para que la aplicación funcione.
Garantizando así la portabilidad y facilidad de despliegue de la solución.

\subsection{Entornos de Julia}
Como lenguaje de programación para ejecutar los modelos generados se escogió Julia. Un lenguaje de programación de alto rendimiento diseñado para la computación técnica y científica.
Se escogió este lenguaje por las siguientes ventajas que ofrece:
\begin{itemize}
    \item Posee un paquete de modelado de optimización llamado \textit{JuMP}, con una sintaxis declarativa que permiten definir variables y restricciones muy similar a una modelación teórica.
    \item Integración sencilla con numerosas herramientas de optimización y soluciona-dores (Ipopt, NLopt, Cbc, HiGHS, SCIP, etc) a través de un paquete llamado \textit{MathOptInterface}.
    \item Compilación Just-In-Time (JIT) sobre LLVM, que proporciona velocidad comparable a C/Fortran en fases de solución y en evaluaciones repetidas de modelos grandes.
    \item Diferenciación automática integrada (ForwardDiff, ReverseDiff), para obtener gradientes y Hessianas exactas en problemas no lineales sin código manual adicional.
\end{itemize}
Dada la amplia gama de problemas que la solución propuesta es capaz de abordar, se ha implementado una infraestructura modular basada en cuatro entornos de ejecución de Julia independientes.
El propósito de esta segmentación es aislar las bibliotecas y dependencias específicas para cada categoría de problema, una práctica recomendada para prevenir conflictos e incompatibilidades entre paquetes de software.
Estos entornos se han especializado de la siguiente manera:
\begin{itemize}
    \item Programación Lineal y Entera Mixta: Un entorno dedicado a problemas LP y MILP que incluye bibliotecas y solvers como \textit{HiGHS} y \textit{Cbc}.
    \item Programación no Lineal y Entera Mixta: Un segundo entorno para problemas NLP y MINLP, equipado con herramientas como \textit{Ipopt}, \textit{NLopt} y \textit{SCIP}.
    \item Satisfacción de Restricciones: Un tercer entorno especializado en problemas CS que contiene la biblioteca \textit{ConstraintSolver}.
    \item Metaheurísticas: Finalmente, un cuarto entorno que agrupa un conjunto de herramientas como \textit{BlackBoxOptim} y \textit{Metaheuristics} basadas en metaheurísticas. Este último está diseñado para la resolución aproximada y eficiente de problemas de alta complejidad, donde se prioriza la rapidez en la obtención de una solución de alta calidad por sobre la garantía de optimalidad.
\end{itemize}

No se incluyeron otras herramientas o paquetes como \textit{Gurobi} o \textit{CPLEX} porque a pesar de ser más eficientes y con mayor rendimiento, son herramientas propietarias que requieren licencias de pago.

\subsection{Entorno de Python}
Como se mencionó anteriormente, el servidor principal de la aplicación está desarrollado en \textit{Python} utilizando el framework \textit{FastApi}.
Pero además, en la aplicación se utilizan otras bibliotecas de Python que son imprescindibles para la solución propuesta.
Entre ellas se encuentran: \textit{google-genai} que es la encargada de interactuar con los modelos de lenguaje y \textit{or-tools}.
Esta última es una biblioteca desarrollada por Google, que proporciona herramientas para resolver problemas de optimización complejos específicos como enrutamiento y planificación.
Se añadió al proyecto como una alternativa para resolver este conjunto de problemas por vías no exactas de manera eficiente.

\subsection{Interfaz gráfica}
Como parte de la solución se desarrolló una interfaz gráfica web utilizando el framework \textit{Next.js}.
Esto permite al usuario interactuar con la aplicación a través de una interfaz tipo chat, que facilita el intercambio de información necesario para obtener la definición del problema de optimización.

\subsection{Modelos de lenguaje}
Como núcleo de cada agente en la implementación se emplearon dos Grandes Modelos de Lenguaje (LLM): \textit{Gemini 2.5 Flash} y \textit{Gemini 2.5 Pro}.
La selección de estos modelos, desarrollados por Google, se fundamentó en un criterio pragmático: la disponibilidad de una Interfaz de Programación de Aplicaciones (API) con un nivel de acceso gratuito, lo cual fue un factor determinante para la viabilidad del proyecto.



\section{Implementación de la solución}
Para la implementación general de cada agente se optó por no realizar \textit{fine-tuning} sobre algún modelo de lenguaje para entrenarlo específicamente en modelar problemas de optimización, no solo debido al coste en recursos computacionales para entrenar los modelos, sino en su mayor parte, por la falta de datos.
Debido a que casi no existen bases de datos que contengan descripciones de problemas de optimización junto a su modelación matemática o código correspondiente.
Y las pocas que se encontraron solo contenían una modelación del problema en \textit{GurobiPy}, un lenguaje de dominio específico de \textit{Python} que se integra con la herramienta \textit{Gurobi}, que como se mencionó anteriormente, requiere una licencia de pago.
En consecuencia se optó por la variante de usar \textit{in-context learning} para trabajar con los modelos de lenguaje.
Esta decisión se apoya en el uso y efectividad de esta técnica en algunos trabajos mencionados en el estado del arte.



\subsection{Modelación}
La modelación de los problemas en formato \textit{FNP} se llevó a cabo mediante el uso de una secuencia de \textit{prompts} diseñados para guiar al modelo en el proceso de estructuración.
Con el objetivo de potenciar el rendimiento del LLM en la generación del modelo de optimización, se implementó la técnica de \textit{few-shot prompting}.
Se le suministró al modelo múltiples ejemplos de descripciones de problemas de optimización, junto con su correspondiente modelación en formato \textit{FNP}.
Adicionalmente, se proporcionaron explicaciones específicas sobre la forma de modelar las restricciones especiales.
Las restricciones especiales que la solución está habilitada para modelar son funciones predefinidas y limitadas, incluyendo restricciones como \textit{all\_different} (todas las variables deben ser diferentes), \textit{at\_least} (deben existir al menos $k$ variables con valor $v$) y \textit{exact} (exactamente $k$ variables deben tener el valor $v$).

El modelo del problema generado es sometido a una revisión mediante un \textit{prompt} reflexivo, el cual habilita al sistema para identificar y corregir posibles errores en la modelación final.
Se optó por incorporar esta técnica basándose en su utilización en trabajos previos citados en el estado del arte.
No obstante, a diferencia de algunas de esas implementaciones, que aplican \textit{prompts} reflexivos de manera iterativa en cada sección de la modelación (variables, restricciones, función objetivo, etc.), en la presente solución se decidió utilizar un único \textit{prompt} reflexivo al culminar el proceso completo de modelación.
Esta elección fue motivada por la necesidad de mitigar el considerable aumento del coste computacional que implica el uso de múltiples \textit{prompts} reflexivos por cada problema.

Como método de validación del modelo generado, solo se aplica una validación sintáctica, se verifica que todas las piezas importantes de la modelación estén presentes como llaves en el archivo \textit{JSON} y se verifica que se pueda parsear el archivo.


\subsection{Generación de código}
A diferencia de \textit{Python}, un lenguaje de programación muy utilizado y que con el cúal se han entrenado a los LLMs ampliamente, lenguajes como \textit{Julia}, más específicos y menos populares, tienden a ser más complicados de generar para los modelos de lenguaje.
Con una mayor probabilidad de que alucinen y cometan errores sintácticos o semánticos al generar código en estos lenguajes.
Para mitigar estos problemas al escoger \textit{Julia} como lenguaje de programación se utilizaron dos estrategias fundamentales:
\begin{itemize}
    \item Se diseñaron diferentes \textit{prompts}, cada uno destinado a una sección de la modelación por separado (variables, parámetros, conjuntos, restricciones y objetivo).
          Estos \textit{prompts} eran usados por el agente de generación de código para interactuar con el modelo de lenguaje y generar un programa con mayor correctitud.
          Cada \textit{prompt} contiene una descripción de la sintaxis específica para esa sección del modelo, así como varios ejemplos del modelo asociado al código resultante.
    \item También se diseñaron \textit{prompts} para resolver cada tipo de problema de optimización, donde se especificaba qué herramientas usar, sintaxis de cada paquete específico con sus configuraciones, así como códigos y sintaxis específicas de cada tipo de problema.
          Por ejemplo: la sintaxis para las restricciones en problemas no lineales era diferente a la utilizada en problemas lineales, así como las herramientas utilizadas y las configuraciones de dichas herramientas.
\end{itemize}

Todo lo anterior se hizo con el objetivo de dar las instrucciones al modelo de la manera más específica posible, con la intención de minimizar la libertad del modelo para generar código, reduciendo así la probabilidad de alucinaciones.
A esto se le suma el uso de la técnica de \textit{prompting} \textit{few-shot} en cada instrucción suministrada al modelo en cuanto a tareas de generación de código.

Para el correcto funcionamiento de esta sección del flujo de solución es fundamental que el modelo genere un código que cumpla una serie de instrucciones específicas.
Para ello se diseñaron una serie de \textit{prompts} que contienen la indicaciones necesarias para que el modelo genere un código que cumpla con los siguientes requisitos:
\begin{itemize}
    \item El código debe incluir una instrucción que limite el tiempo que puede ejecutarse la herramienta o solucionador, evitando que el programa pueda ejecutarse indefinidamente, o que sobrepase el límite impuesto por el usuario.
    \item Debe además evitar que la herramienta utilizada imprima cualquier información en consola, ya que esto podría interferir con la salida esperada del programa.
    \item Las respuestas del problema obtenidas al ejecutar el programa deben imprimirse en consola en formato \textit{JSON} para que el agente pueda capturar ese resultado e interpretarlo.
    \item En el caso específico de que el problema sea MINLP y la función objetivo sea no lineal se debe usar una técnica llamada reformulación epígrafo o hipógrafo. Debido a que la única herramienta capaz de resolver este tipo de problemas es \textit{SCIP}, que soporta restricciones no lineales pero no funciones objetivo no lineales.
          Por lo tanto se crea una nueva función objetivo lineal y se restringe con la anterior función objetivo no lineal. Y se optimiza en el sentido contrario a la optimización original.
\end{itemize}
\chapter{Experimentos y Resultados}\label{chapter:experiments}


\section{Diseño de los experimentos}
Considerando el objetivo central de este trabajo, la evaluación del desempeño del sistema se centrará en su capacidad para resolver problemas prácticos de extremo a extremo.
El criterio fundamental será validar si la solución logra encontrar un resultado final correcto, partiendo únicamente de la descripción inicial del problema planteado.
También se evaluó el desempeño de cada uno de los componentes y técnicas utilizadas en la solución: que tanto aportan a la correctitud del resultado final o a diferentes parámetros como la ejecutabilidad y la extracción correcta de los datos del problema.

\subsection{Conjuntos de datos}
Se seleccionaron varios conjuntos de datos que contienen problemas de optimización formulados en lenguaje natural para la evaluación de la solución propuesta.
Estos conjuntos de datos fueron seleccionados debido a que son comúnmente utilizados en la literatura estudiada para evaluar soluciones con el mismo objetivo: resolver problemas de optimización a partir de descripciones en lenguaje natural.
A continuación se presentan los conjuntos de datos seleccionados y una breve descripción de los mismos:
\begin{itemize}
      \item \textbf{NLP4LP:} Un conjunto de datos que se introduce en el trabajo presentado por~\cite{ahmaditeshnizi2025optimus03usinglargelanguage}, contiene un total de 241 problemas de optimización. De los cuales 210 son problemas LP y 31 son problemas MILP\@.
            Los problemas de este conjunto de datos en promedio presentan un longitud en su descripción en lenguaje natural de 536 palabras, tienen en promedio 2 variables, 3 familias de restricciones y 5 parámetros.
            Cada problema del conjunto contenía además su modelación en \textit{GurobiPy} y los valores óptimos de las variables y la función objetivo para las evaluaciones.
      \item \textbf{MAMO easy:} En el trabajo de~\cite{huang2025llmsmathematicalmodelingbridging}, se presentan varios \textit{benchmarks} para evaluar las capacidades de los LLMs en la generación de modelos matemáticos, tanto de optimización como ecuaciones diferenciales a partir de descripciones en lenguaje natural.
            Uno de estos \textit{benchmarks} es el conjunto de datos MAMO easy, que contiene un total de 652 problemas de optimización, de los cuales 640 son problemas MILP y 12 son problemas LP\@. Con una longitud promedio en su descripción de 1045 palabras. Además cada problema cuenta con un aproximado de 3 variables, 4 familias de restricciones y 4 parámetros en promedio.
            Cada problema contenía solo la solución óptima de la función objetivo, pero no los valores óptimos de las variables.
      \item \textbf{MAMO complex:} Este es otro conjunto de datos presentado en el mismo trabajo junto a MAMO easy, este set de problemas se considera más complejo y desafiante que los de MAMO easy.
            Consta de 211 problemas que abarcan temáticas como flujo, enrutamiento, inventario, entre otros. Aunque este conjunto no contiene problemas no lineales, el tamaño de sus problemas es mayor, con un promedio de 32 variables por problema.
      \item \textbf{Complex OR:} Un conjunto de datos presentado en el trabajo de~\cite{xiao2024chainofexperts}, con una variedad de problemas recopilados de situaciones reales, o de libros de texto de investigación operativa.
            Este conjunto contiene un total de 18 problemas de optimización, de los cuales 5 son LP y 13 son MILP\@. Con una longitud promedio en su descripción de 1001 palabras, y aproximadamente 5 variables, 10 parámetros y 4 familias de restricciones en promedio.
      \item \textbf{Opti-Bench:} Este grupo de problemas se presentaron en el trabajo de~\cite{optibench2024}, posee una gran cantidad y variedad de problemas de optimización que simulan situaciones reales.
            El conjunto de datos contiene un total de 605 problemas, entre ellos 102 son LP, 330 son MILP, 86 son NLP y 85 son MINLP\@. Además algunos de los problemas contienen datos tabulares, simulando escenarios y problemas prácticos.
            Con una longitud promedio en su descripción de 680 palabras. La cantidad promedio de variables varia en dependencia del tipo problema, teniendo desde 3 para problemas LP hasta 6 para MINLP\@.
\end{itemize}

Debido a la gran cantidad de problemas en el conjunto de datos \textit{Opti-Bench} y la variedad de tipos de problemas que contiene, se seleccionó como el principal conjunto de datos para la evaluación de la propuesta.
Otros conjuntos de problemas como \textit{LPWP} presentados en la competencia \textit{NL4Opt} (\cite{ramamonjison2023nl4optcompetitionformulatingoptimization}) no fueron considerados para la evaluación debido a dos motivos.
El conjunto se construyó con el objetivo de evaluar soluciones que modelaran un problema de optimización, no está diseñado para evaluar soluciones que ejecuten todo el flujo para resolver un problema.
Además, como se menciona en el trabajo de~\cite{optibench2024} este conjunto esta compuesto de problemas demasiado sencillos para los modelos de lenguaje actuales, debido a que cuando se construyó el conjunto \textit{LPWP} los modelos de lenguaje no tenían capacidades ni remotamente cercanas a las que presentan hoy en días.


\subsection{Configuración de los experimentos}
La evaluación de la propuesta requirió un diseño estructurado para la experimentación, permitiendo la medición ablacionada de cada componente clave de la propuesta.
Se generó una serie de configuraciones diferentes para evaluar dichos componentes claves, cada configuración depende de una serie de parámetros que representan la activación de cada componente.
En esta subsección se detalla la metodología utilizada para construir las configuraciones de los experimentos.
Además se describen una serie de experimentos extras, que evalúan la relevancia de otra serie de técnicas utilizadas en la solución.
A continuación se muestran los principales parámetros que definen cada configuración de los experimentos:
\begin{itemize}
      \item \textbf{Think:} Este parámetro permite seleccionar si usar un modelo más avanzado de razonamiento (\textit{Gemini 2.5 Pro}) o el modelo base (\textit{Gemini 2.5 Flash}).
      \item \textbf{Reflexive:} Permite activar o desactivar la retroalimentación reflexiva en el agente de modelación.
      \item \textbf{Transform:} Este parámetro representa si el agente conversacional debe realizar modificaciones en la descripción del problema, con el objetivo de hacerlo más entendible y explícito como se menciona en la subsección~\ref{subsec:conv-agent}.
\end{itemize}

Debido a la gran cantidad de problemas y a la limitada capacidad de uso de la \textit{API} del modelo \textit{Gemini 2.5 Pro} comparada con la del modelo \textit{Gemini 2.5 Flash}, se decidió solo usar este modelo en el conjunto de datos \textit{Opti-Bench}.
Sobre este conjunto se ejecutaron 6 rondas de experimentos, en cada ronda se ejecutaron todos los problemas del conjunto con una configuración diferente.
Estas configuraciones se muestran en la tabla~\ref{tab:configurations}.
\begin{table}[h]
      \centering
      \begin{tabular}{cccc}
            \toprule
            \textbf{Ronda} & \textbf{Think} & \textbf{Reflexive} & \textbf{Transform} \\
            \midrule
            1              & No             & Si                 & Si                 \\
            2              & No             & Sí                 & No                 \\
            3              & No             & No                 & Sí                 \\
            4              & Sí             & Sí                 & Sí                 \\
            5              & Sí             & Sí                 & No                 \\
            6              & Sí             & No                 & Sí                 \\
            \bottomrule
      \end{tabular}
      \caption{Configuraciones de los experimentos sobre el conjunto \textit{Opti-Bench}}
      \label{tab:configurations}
\end{table}

Las configuraciones donde todos los parámetros están activados representan la configuración real de la solución propuesta.
Por otra parte las configuraciones donde existe algún parámetro desactivado permiten evaluar el aporte de ese parámetro a la solución final, en otras palabras se mide que tanto disminuye el rendimiento de la solución al desactivar ese parámetro.



Para el resto de conjuntos de problemas solo se variaron los parámetros \text{Transform} y \text{Reflexive}, ya que no se usó el modelo \textit{Gemini 2.5 Pro} en estos experimentos. Se realizaron 3 rondas de experimentos sobre cada conjunto, con las configuraciones mostradas en la tabla~\ref{tab:configurations-other}.
\begin{table}[h]
      \centering
      \begin{tabular}{ccc}
            \toprule
            \textbf{Ronda} & \textbf{Reflexive} & \textbf{Transform} \\
            \midrule
            1              & Si                 & Si                 \\
            2              & Si                 & No                 \\
            3              & No                 & Si                 \\
            \bottomrule
      \end{tabular}
      \caption{Configuraciones de los experimentos en el resto de conjuntos}
      \label{tab:configurations-other}
\end{table}


Se ejecutó además, una serie de experimentos extras para evaluar y comparar otros aspectos más específicos de las técnicas y estrategias utilizadas.
En primer lugar se realizó una ronda extra de experimentos sobre cada conjunto de datos sin utilizar los \textit{prompts} que daban instrucciones y ejemplos específicos sobre como generar código de \textit{Julia} al agente de generación de código.
De esta forma se evaluó la efectividad de esta serie de instrucciones y ejemplos en la generación correcta de un código con el que un modelo de lenguaje puede no estar tan familiarizado.
También se realizaron dos rondas extras sobre el conjunto principal \textit{Opti-Bench}, la primera fue usando un modelo incluso menos potente (\textit{Gemini 2.0 Flash}), y la segunda fue usando \textit{Gemini 2.5 Pro} como optimizador, sin usar la solución propuesta, en otras palabras: se le pidió al modelo directamente que resuelva el problema.
Esto se hizo con el objetivo de comparar el desempeño de los LLMs por si solos para resolver problemas de optimización en contraste con la solución de un enfoque basado en integrar herramientas externas junto a modelos de lenguaje aunque estos sean menos potentes.



\subsection{Ejecución y revisión de los experimentos}
La fase de experimentación se estructuró en dos procesos fundamentales: la ejecución automatizada y validación de los experimentos, y la posterior revisión manual de los resultados obtenidos.
A continuación, se detalla la metodología implementada en cada etapa, destacando su importancia para asegurar la fiabilidad de los resultados.
Para la ejecución de cada conjunto de datos se crearon \textit{scripts} utilizando \textit{Python} para automatizar el proceso y poder cambiar fácilmente los parámetros de configuración.
Dichos \textit{scripts} se encargaban de enviar cada problema a la aplicación y al recibir la respuesta corroborar su correctitud, para luego almacenar los resultados en un archivo \textit{JSON}.
Verificar la correctitud de cada problema dependía de cada base de datos, en la mayoría solo se contaba con el valor óptimo de la función objetivo, por lo que se consideraba correcto si el valor obtenido por la solución coincidía con el valor óptimo reportado.
En los conjuntos donde además se contaba con las variables óptimas, estas también debían coincidir para considerar el problema como resuelto correctamente.
Debido a posibles errores numéricos y la naturaleza de algunos problemas no lineales, se tuvo en cuenta un margen de error de hasta un 0.1\% en los valores de la función objetivo y las variables.

Después de ejecutar los experimentos se realizó una revisión manual sobre los resultados obtenidos de cada problema debido a varios motivos:
\begin{itemize}
      \item En algunos problemas, el valor óptimo reportado podría coincidir con el valor óptimo obtenido, sin embargo podría haber discrepancias en las variables.
            Indicando uno de dos casos: que existen varias combinaciones de valores de las variables que llevan al mismo valor óptimo, o por otro lado, que los valores encontrados incumplen alguna restricción que se modeló incorrectamente.
      \item En algunos problemas, el valor óptimo reportado podría ser incorrecto.
      \item Los valores óptimos de variables booleanas podrían aparecer representadas de diferentes formas: como 0 y 1 en algunos casos y como \textit{True} y \textit{False} en otros.
      \item En los conjuntos de datos que solo contenían el valor óptimo de la función objetivo, existía la posibilidad de que el valor óptimo encontrado coincidiera, pero los valores de las variables encontradas incumplieran alguna restricción que se omitió en la modelación o se modeló incorrectamente
\end{itemize}

Durante la revisión manual se encontraron varios problemas que presentaban algunas de las situaciones mencionadas anteriormente.
Para la mayoría de estos casos se modeló el problema de manera manual y se corroboraron los resultados.
En los problemas que reportaban un valor óptimo incorrecto, se encontró que la mayoría de estos errores se debían a que el problema no presentaba soluciones factibles, pero aún así se reportaba un valor óptimo.
Esta revisión de los experimentos se realizó con el objetivo de asegurar la precisión de los resultados reportados en este trabajo.

\section{Resultados}
En esta sección se presentan los resultados obtenidos en los experimentos realizados.
Primero se muestran los resultados generales de la solución, y se realiza un análisis comparativo con otras soluciones similiares.
Luego se presentan los resultados obtenidos de las diferentes configuraciones presentadas en la sección anterior.
Finalmente, se presentan otros resultados y métricas relevantes obtenidas durante la evaluación de la solución propuesta

\subsection{Resultados Generales}
Se recopiló los resultados presentados por otras soluciones mencionadas en el estado del arte sobre los conjuntos de datos presentados, con el objetivo de realizar una comparación directa con los resultados obtenidos por la propuesta.
La principal métrica a comparar es la precisión, que representa el porcentaje de problemas resueltos correctamente en base al total de problemas.
En la mayoría de los conjuntos seleccionados se obtuvieron mejores resultados con respecto a los presentados en el estado del arte, exceptuando solamente el conjunto \textit{MAMO easy}.
En el resto de los conjuntos, se superó la cantidad de problemas resueltos correctamente por otros trabajos por un margen de al menos un 9\%.
En la tabla~\ref{tab:general_results} se presenta una comparación más detallada entre los mejores resultados obtenidos por la solución propuesta y los resultados reportados por otros trabajos en cada uno de los conjuntos de datos escogidos.

\begin{table}[h]
      \centering
      \makebox[\textwidth]{
            \begin{tabular}{cccccc}
                  \toprule
                                     & \textbf{Opti-bench} & \textbf{NLP4LP} & \textbf{MAMO easy} & \textbf{MAMO complex} & \textbf{Complex OR} \\
                  \midrule
                  \textbf{Optim AI}  & 87.4                & 88.1            & \textendash        & \textendash           & \textendash         \\
                  \midrule
                  \textbf{Optimus}   & \textendash         & 80.6            & \textendash        & \textendash           & 66.7                \\
                  \midrule
                  \textbf{ORLM}      & \textendash         & 72.9            & 82.3               & 37.9                  & \textendash         \\
                  \midrule
                  \textbf{LLMOPT}    & \textendash         & 83.8            & \textbf{97.0}      & 68.0                  & 72.7                \\
                  \midrule
                  \textbf{CoE}       & \textendash         & 49.2            & \textendash        & \textendash           & 31.4                \\
                  \midrule
                  \textbf{Propuesta} & \textbf{96.8}       & \textbf{100}    & 94.6               & \textbf{83.0}         & \textbf{88.8}       \\
                  \bottomrule
            \end{tabular}
      }
      \caption{Comparación de resultados entre la solución y el estado del arte}
      \label{tab:general_results}
\end{table}


\subsection{Otros resultados y métricas}
En la subsección anterior se presentaron los mejores resultados obtenidos por la solución usando la configuración completa.
En esta subsección se analiza como en algunos casos la configuración usada en la solución varía mucho el rendimiento obtenido.
En la tabla~\ref{tab:results_different_configs} se muestra inicialmente una comparación de los resultados de variar los parámetros \textit{Reflexive} y \textit{Transform} en diferentes experimentos sobre los conjuntos de datos.
En estos experimentos se usenó el modelo \textit{Gemini 2.5 Flash}, y se muestran los resultados obtenidos con la solución propuesta, y los resultados obtenidos al desactivar cada uno de los parámetros \textit{Reflexive} y \textit{Transform}.

\begin{table}[h]
      \centering
      \begin{tabular}{cccc} % l para las etiquetas, r para los números
            \toprule
                                  & \textbf{Propuesta} & \textbf{Sin reflexive} & \textbf{Sin transform} \\
            \midrule
            \textbf{Opti-bench}   & 92.01              & 78.1                   & 89.8                   \\
            \midrule
            \textbf{NLP4LP}       & 100                & 100                    & 100                    \\
            \midrule
            \textbf{MAMO easy}    & 94.6               & 94.0                   & 94.2                   \\
            \midrule
            \textbf{MAMO complex} & 83.0               & 75.3                   & 75.8                   \\
            \midrule
            \textbf{Complex OR}   & 88.8               & 88.8                   & 88.8                   \\
            \bottomrule
      \end{tabular}
      \caption{Resultados obtenidos al usar diferentes configuraciones en la solución.}
      \label{tab:results_different_configs}
\end{table}


Como se mencionó anteriormente, en el conjunto de datos \textit{Opti-Bench} se utilizó el cambio de parámetro \textit{Think}.
Permitiendo así comparar el rendimiento de la solución usando dos modelos de lenguaje con diferentes capacidades (\textit{Gemini 2.5 Flash} y \textit{Gemini 2.5 Pro}).
En la tabla~\ref{tab:think_experiments} se muestra la comparación entre los resultados obtenidos usando cada uno de estos modelos como base de la solución propuesta.
Se aprecia una mejora relevante al usar el modelo \textit{Gemini 2.5 Pro} en los resultados de la solución.

\begin{table}[h]
      \centering
      \begin{tabular}{cccc}
            \toprule
                                      & \textbf{Propuesta} & \textbf{Sin reflexive} & \textbf{Sin transform} \\
            \midrule
            \textbf{Gemini 2.5 Flash} & 92.01              & 78.1                   & 89.8                   \\
            \midrule
            \textbf{Gemini 2.5 Pro}   & 96.8               & 93.5                   & 94.2                   \\
            \bottomrule
      \end{tabular}
      \caption{Resultados obtenidos en el conjunto de datos \textit{Opti-bench} usando diferentes modelos de lenguaje como base de la solución.}
      \label{tab:think_experiments}
\end{table}


Además de la precisión, la evaluación de la solución incluyó la medición de otros aspectos relevantes.
Entre ellos se encuentran la ejecutabilidad del código generado —definida como la cantidad de problemas que se ejecutaron exitosamente sin producir errores antes de alcanzar el límite de ciclos de depuración—, la cantidad de problemas que requirieron ciclos de depuración, y el promedio de ciclos de depuración necesarios por problema.
Estas métricas no solo proporcionan una medida de la calidad del codigo generado, sino que también se emplearon para validar la efectividad de las instrucciones específicas de \textit{Julia} proporcionadas en los \textit{prompts} al agente de generación de código.
La Tabla~\ref{tab:code_metrics} detalla una comparativa usando estas métricas de dos experimentos diferentes: uno usando las instrucciones específicas de \textit{Julia} y otro omitiéndolas.
\begin{table}[h]
      \centering
      \makebox[\textwidth]{
            \begin{tabular}{cccccc}
                  \toprule
                                           & \textbf{Opti-bench} & \textbf{MAMO easy} & \textbf{MAMO complex} & \textbf{Complex OR} \\
                  \midrule
                  \textbf{Ejecutabilidad}  & 100                 & 100                & 99.1                  & 100                 \\
                  \midrule
                  \textbf{Depuración}      & 10                  & 8.4                & 12.7                  & 5.5                 \\
                  \midrule
                  \textbf{Ciclos promedio} & 1.73                & 1.4                & 1.9                   & 1.1                 \\
                  \midrule
                  \multicolumn{5}{c}{\textbf{Sin instrucciones específicas de Julia}}                                               \\
                  \midrule
                  \textbf{Ejecutabilidad}  & 98.0                & 100                & 98.58                 & 100                 \\
                  \midrule
                  \textbf{Depuración}      & 17.4                & 8.4                & 14.9                  & 22.2                \\
                  \midrule
                  \textbf{Ciclos promedio} & 2.33                & 1.4                & 2.1                   & 1.7                 \\
                  \bottomrule
            \end{tabular}
      }
      \caption{Resultados adicionales obtenidos con la solución propuesta.}
      \label{tab:code_metrics}
\end{table}



Otra comparación relevante que se realizó fue verificar la capacidad de resolver problemas de optimización de un modelo de lenguaje razonador por si solo, en contraste con la solución propuesta usando un modelo mucho menos potente.
Como se mencionó anteriormente se escogieron los modelos \textit{Gemini 2.5 Pro} y \textit{Gemini 2.0 Flash} para esta comparación.
Se obtuvo que la solución propuesta usando el modelo menos potente superó en casi un 4\% la precisión alcanzada con el modelo más avanzado usado como optimizador, específicamente se alcanzó una precisión de un 85.45 y un 88.8 respectivamente.



\section{Discusión de los resultados}\

Los resultados experimentales expuestos en la sección anterior proporcionan una visión general del desempeño de la solución propuesta.
Sin embargo, el propósito de esta sección es profundizar en el análisis e interpretación de dichos resultados.
Que implicaciones tienen los valores obtenidos en los experimentos, y qué conclusiones se pueden extraer sobre la efectividad de las técnicas y estrategias empleadas en este trabajo.


En la tabla~\ref{tab:configurations} se observa como en conjuntos como \textit{NLP4LP}, \textit{MAMO easy} y \textit{Complex OR} desactivar alguno de los parámetros no afecta prácticamente el rendimiento de la solución, mostrando una disminución de menos del 1\%.
Esto podría deberse a que son conjuntos de problemas relativamente más sencillos, por lo que el uso de estas técnicas no es imprescindible para obtener buenos resultados.
Sin embargo, en conjuntos más complejos como \textit{Opti-Bench} y \textit{MAMO complex} se observa una diferencia considerable al desactivar alguno de los parámetros.
Entre los dos parámetros analizados, el que representó un mayor impacto según los resultados de los experimentos fue el que activa la retroalimentación reflexiva, mostrando en el conjunto \textit{Opti-Bench} una disminución de casi 14\% en la precisión al desactivar este parámetro durante la ejecución de los experimentos.
En cuanto a la transformación de la descripción del problema, la disminución que representó en el rendimiento es mucho menor, en comparación con el parámetro \textit{reflexive}.
Sin embargo en algunos problemas específicos se pudo observar que la transformación de la descripción evitó errores en la identificación de la naturaleza de las variables.

En los experimentos que se muestran en la tabla~\ref{tab:think_experiments} se aprecia como desactivar algún parámetro tiene menos efecto si el modelo de lenguaje es más potente.
Esto podŕia deberse a que al ser un modelo más potente su cadena de pensamiento es mayor, cubriendo posibles errores que cometería en primer lugar, evitando tener que arreglarlos después en la revisión reflexiva.
Sin embargo debido al tamaño y complejidad limitadas en los problemas de nuestros conjuntos de datos, no es posible comprobar si al aumentar el tamaño del problema junto al tamaño del modelo de lenguaje los parámetros vuelven a recobrar más influencia en la precisión obtenida.




En términos de ejecutabilidad se determinó que en cada conjunto de problemas se mantuvo prácticamente constante, incluso en ausencia de instrucciones específicas como se aprecia en la tabla~\ref{tab:code_metrics}.
Las únicas métricas que experimentaron una leve variación fueron la cantidad de problemas que requirieron al menos una iteración de depuración y el promedio de ciclos de corrección necesarios.
Esta estabilidad en la ejecutabilidad puede atribuirse a la robustez inherente del modelo de lenguaje para arreglar código erróneo a partir de un mensaje de error.
Pese a que se incrementa el número de depuraciones requeridas, el modelo demuestra ser lo suficientemente potente como para converger en un código ejecutable después de un número limitado de iteraciones.
Aunque incluso sin las instrucciones específicas de \textit{Julia} la solución puede resolver los problemas, no significa que no sean necesarias, ya que aportan al tiempo de solución y el gasto de recursos de los modelos de lenguaje.
Así que con estas instrucciones la solución termina siendo más eficiente.


La solución propuesta está diseñada como se menciona en capítulos anteriores, para tener en cuenta un tiempo límite establecido por un usuario, permitiendo así cumplir con las necesidades del usuario y resolver el problema en tiempo, o al menos dar una solución factible.
Sin embargo se decidió que el parámetro de tiempo no era algo a tener en cuenta en la experimentación de la solución.
Esto se debe al tamaño y dificultad limitada que presentaban los problemas de los conjuntos de datos, cada problema se resolvía en cuestión de pocos segundos, dejando muy poco espacio para experimentar con el parámetro del tiempo en este trabajo.




\subsection{Principales errores}
El análisis detallado de los resultados permitió identificar los errores más recurrentes cometidos por la solución durante el proceso de resolución de problemas.
El error principal detectado fue la identificación incorrecta de la naturaleza de las variables.
Específicamente, en ciertos escenarios, la solución clasificaba erróneamente variables discretas como continuas.
Este tipo de confusión se observó predominantemente en contextos que involucraban variables relacionadas con alimentos en problemas de nutrición o con hectáreas de tierra en modelos de agricultura.
Un segundo error, menos frecuente pero también significativo, fue la clasificación incorrecta de una restricción de desigualdad como una restricción de igualdad.
Estos fallos se manifestaron principalmente en problemas de oferta y demanda.
La aparición de frases imperativas como ``se debe suplir la demanda de cada lugar'' o ``se planean invertir x cantidad de recursos'' tendía a inducir al modelo a interpretar la condición como una restricción de igualdad estricta, en lugar de una restricción de desigualdad.

\backmatter

\begin{conclusions}
    El presente trabajo de tesis abordó el desafío fundamental de resolver problemas de optimización a partir de descripciones en lenguaje natural.
    Se desarolló con éxito una solución que funciona como una interfaz lingüística para dichos problemas, acorde a los objetivos del trabajo.
    Se utilizó como base la investigación realizada sobre la literatura para el diseño de la arquitectura, el formato de modelación formal propuesto y otras técnicas complementarias que se usaron en la solución.
    Se implementó usando una estructura multiagente, integrando modelos de lenguaje con herramientas y \textit{solvers} externos.
    Se agregó además un mecanismo de conversación multi-ronda para asistir la falta de conocimientos del usuario, acorde con los objetivos de este trabajo.
    Se cumplió con el objetivo de evaluar la solución propuesta mediante experimentos usando conjuntos de datos seleccionados del estado del arte.
    Donde se alcanzó resultados superiores en términos de precisión con respecto a trabajos previos.
    Se realizó un estudio ablacionado adicional para analizar el impacto de las técnicas y estrategias utilizados en la solución.
    Todo lo anterior valida el cumplimiento de los objetivos generales y específicos planteados al inicio de este trabajo de tesis.
\end{conclusions}

\begin{recomendations}
    Uno de las principales limitantes existentes en este trabajo y en otros similares encontrados en el estado del arte, es una pobre evaluación en problemas reales prácticos.
    A pesar de que los conjuntos de problemas escogidos son variados y cubren diferentes tipos de problemas de optimización, estan lejos de representar la complejidad y tamaños que se pueden encontrar en problemas de situaciones reales, con miles de variables y restricciones.
    Por lo tanto, se recomienda realizar evaluaciones en problemas reales de mayor tamaño y complejidad, en la medida de lo posible, con el objetivo de validar el desempeño de la solución en escenarios más desafiantes y representativos.
    También se sugiere explorar distintas ideas para corregir los errores más comunes que se mencionan en este trabajo que cometen los modelos a la hora de lidiar con problemas de optimización.
    Finalmente, se recomienda investigar alternativas para mejorar la eficacia de la solución propuesta, en preparación para su posible evaluación en problemas reales de mayor escala.
\end{recomendations}

\include{BackMatter/Bibliography}

\end{document}