\chapter*{Introducción}\label{chapter:introduction}
\addcontentsline{toc}{chapter}{Introducción}
% Optimization background and history
La optimización ha sido una disciplina clave en numerosos campos de la ciencia y la ingeniería a lo largo de la historia.
Aunque la idea de maximizar o minimizar una función ha sido fundamental en el cálculo y la física desde sus inicios, la optimización no se consolidó como una disciplina formal hasta épocas más recientes.
Sus primeras aplicaciones prácticas a gran escala surgieron durante la Segunda Guerra Mundial para resolver asuntos de logística y planificación militar, dando origen a la investigación de operaciones (Operation Research, en inglés)~(\cite{Petropoulos_2023}).
Este campo integraba matemáticas, estadística y experimentación para optimizar operaciones militares y fue pionero en el uso de métodos cuantitativos para la toma de decisiones en organizaciones.
Posteriormente, la inspiración en procesos naturales, aleatorios o en el comportamiento humano dio lugar a los algoritmos metaheurísticos.
Estos métodos ofrecieron nuevas vías para resolver problemas de optimización, permitiendo alcanzar soluciones aproximadas al óptimo en tiempos de cómputo considerablemente menores~(\cite{Tomar2023metaheuristics}).
Este avance representó un salto cualitativo en la capacidad de resolver problemas complejos del mundo real que, hasta ese momento, eran intratables con métodos exactos.
% Growing interest in optimization in recent years and machine learning influence
Dada la relevancia de los problemas de optimización, el interés en este campo ha crecido considerablemente.
Un estudio revela que, en las últimas tres décadas, la producción científica relacionada con este tema —incluyendo investigaciones, artículos y avances importantes— se ha incrementado de forma exponencial~(\cite{weinand2021research}).
El surgimiento del machine learning como herramienta para abordar problemas complejos y de alta dimensionalidad ha sido un catalizador fundamental en este desarrollo.
Los métodos de aprendizaje automático han demostrado ser capaces de modelar relaciones complejas, expandiendo significativamente el espectro de problemas de optimización que pueden resolverse~\cite{bengio2020machinelearningcombinatorialoptimization}.
% Solving optimization problems nowadays
Actualmente, existe una gran cantidad de herramientas y programas de software para resolver problemas de optimización.
Cada una de estas herramientas se especializa en uno o varios tipos de problemas existentes (programación lineal, entera, no lineal, optimización robusta, etc.).
Han alcanzado un nivel de avance y eficiencia que les permite manejar problemas relativamente grandes y complejos en tiempos razonables.
Hoy en día, la resolución de un problema de optimización complejo consta de tres pasos fundamentales~(\cite{zhang2024solvinggeneralnaturallanguagedescriptionoptimization}).
Primero, un experto en el dominio del problema debe analizar los requerimientos y objetivos prácticos, convirtiéndolos en una descripción formal.
Después, se extrae la información relevante de dicha descripción y se representa en un lenguaje de modelado, como Python, AMPL o Julia.
Finalmente, se utilizan las herramientas de software mencionadas anteriormente para obtener la solución.
% The LLM usage in optimizations problems
La ambición de reducir la brecha entre el lenguaje natural y el modelado algebraico o algorítmico de un computador no es nueva.
De hecho, ya en 1996, Eugene C. Freuder describió la idea de que “el usuario especifique el problema y el ordenador lo resuelva” como el “Santo Grial” de la programación de restricciones~(\cite{tsouros2023holygrail20natural}).
Los primeros intentos para alcanzar este objetivo se manifestaron en sistemas expertos que buscaban asistir en la interacción con modelos matemáticos usando lenguaje natural~(\cite{Dantzig1951}).
Sin embargo, debido a su inherente complejidad, el avance en el procesamiento del lenguaje natural fue un proceso lento.
Aunque hasta hace relativamente poco se habían logrado progresos notables en tareas como la traducción de textos, el análisis sintáctico y el reconocimiento de entidades, la capacidad de analizar un problema complejo con su contexto, ambigüedades e ideas implícitas parecía una meta lejana.

A partir de 2018, la aparición de los modelos de lenguaje preentrenados (pre-trained language models, en inglés) como BERT y GPT-2 supuso un cambio de paradigma, logrando un desempeño sobresaliente en tareas de procesamiento y generación de lenguaje.
Estos modelos fueron escalando en tamaño hasta convertirse en los hoy conocidos como Large Language Models (LLM). Al aumentar el número de parámetros y el tamaño de sus corpus de entrenamiento, estos modelos no solo mejoraron sus capacidades para entender, generar y manipular textos, sino que también desarrollaron «habilidades emergentes» que los modelos más pequeños no presentaban~(\cite{zhao2025surveylargelanguagemodels}).
Estas habilidades se manifiestan como una mejora drástica en el desempeño al emplear técnicas de prompting, tales como in-context learning y few-shot learning.
\newpage

\subsection*{Problema}

A pesar de los avances en técnicas y herramientas desarrollados recientemente en el campo de la optimización, hay un gran
problema que todavía persiste: estas herramientas requieren una modelación formal del problema para poder resolverlo. Sin embargo,
los problemas de la vida real no suelen seguir este formato, con frecuencia, describir con exactitud el objeto de optimización resulta complicado incluso para expertos.
Esta dificultad proviene, en gran medida, de la complejidad intrínseca de muchos casos prácticos. Es habitual que incluyan un elevado número de variables, restricciones
implícitas o interrelaciones complejas.
Por lo tanto, en la práctica, se necesita conocimientos especializados en el campo de la optimización y en el uso de dichas herramientas, para poder entender el problema y llevarlo
a una modelación formal.

Este proceso es generalmente costoso en cuestiones de tiempo y recursos, así como demandar la intervención de profesionales con experiencia. Debido a esto existe una barrera entre empresas, negocios o
instituciones, que no cuentan con recursos o personal experto en este campo, y la posibilidad de acceder a las tecnologías de optimización. Siendo este acceso un aspecto
significativo para facilitar la toma de decisiones, mejorar la eficiencia de sus procesos y solucionar problemáticas complejas.


\subsection*{Motivación}
Con este trabajo de tesis se realiza una propuesta para aumentar la accesibilidad a las tecnologías de optimización. Permitiendo a empresas, negocios, investigadores o cualquier persona,
solucionar problemas prácticos de su contexto de forma sencilla e intuitiva. En general, que usuarios no expertos, puedan resolver problemas de optimización a partir de una conversación.
Se busca que a través de un pequeño intercambio, se pueda extraer toda la información del problema y resolverlo automáticamente, dando en todo momento retroalimentación sobre el proceso. Esto permite
solucionar el problema incluso si el usuario no posee un entendimiento fundamental del mismo.



\subsection*{Antecedentes}
El empleo de los LLMs para permitir que usuarios no especializados aborden problemas reales complejos ha sido objeto de investigaciones previas en la Facultad de Matemáticas y Computación de la Universidad de
La Habana. En este contexto, se han desarrollado asistentes virtuales impulsados por LLMs con el propósito de facilitar la comprensión y el acceso a la información de la legislación cubana, así como de optimizar
la interacción y la extracción de datos de los complejos cuadros tabulares incluidos en los Anuarios Estadísticos de Cuba.


\subsection*{Objetivos}

El objetivo general de este trabajo es el diseño e implementación de una propuesta de solución al problema planteado.
Una solución que no dependa de conocimientos en optimización por parte del usuario, y que tampoco necesite que el problema esté modelado formalmente para resolverlo.
Por lo tanto se propone como objetivo que la solución trabaje con la forma más natural y simple de un problema: su descripción en lenguaje natural.
La propuesta de solución debe ser capaz de extraer la información relevante a partir de la descripción de un problema: objetivos, restricciones, variables, etc. Usando esa información para modelar formalmente el problema y resolverlo de forma automática.
Para cumplir con este objetivo general, se plantean los siguientes objetivos específicos:
\begin{itemize}
    \item Realizar un análisis sobre el estado del arte enfocado en la modelación y solución de problemas de optimización a partir de lenguaje natural.
    \item Analizar las distintas tecnologías, herramientas y lenguajes de modelado utilizados para resolver los problemas de optimización actuales.
    \item El diseño de una interfaz visual para que el usuario pueda interactuar de forma sencilla con la solución implementada.
    \item El diseño de un mecanismo que ayude a suplir la falta de experiencia de los usuarios en problemas de optimización, guíandolos durante la descripción del problema por parte de cada usuario.
    \item La implementación de un sistema de comprobación y validación de la solución propuesta, que permita asegurar la calidad de los resultados obtenidos.
    \item Evaluar la efectividad de la propuesta, y su rendimiento comparado a otras soluciones similares.
\end{itemize}


\subsection*{Estructura de la tesis}
El resto del trabajo de tesis está estructurado de la siguiente manera.
En el Capítulo~\ref{chapter:state-of-the-art} se realiza una revisión de la literatura, analizando los diferentes enfoques y técnicas utilizados para modelar y resolver problemas de optimización a partir de descripciones en lenguaje natural.
A continuación, en el Capítulo~\ref{chapter:proposal} se presenta la arquitectura de la solución desarrollada, detallando el sistema multi-agente propuesto y las estrategias que sigue para interpretar y resolver cada problema.
El Capítulo~\ref{chapter:implementation} profundiza en los aspectos técnicos de la solución, describiendo la infraestructura, los entornos de programación, las herramientas y los modelos de lenguaje que se emplearon para su construcción.
Posteriormente, en el Capítulo~\ref{chapter:experiments} se describe el diseño experimental, los conjuntos de datos utilizados para la evaluación y se presentan y discuten los resultados obtenidos, comparando el rendimiento de la propuesta con otros trabajos del área.
Finalmente, se exponen las Conclusiones que resumen los hallazgos del trabajo y las Recomendaciones para futuras líneas de investigación.