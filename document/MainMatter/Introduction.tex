\chapter*{Introducción}\label{chapter:introduction}
\addcontentsline{toc}{chapter}{Introducción}
% Optimization background and history
La optimización ha sido una disciplina clave en muchos campos de la ciencia y la ingeniería a lo largo de la historia. Desde hace muchos años,
la idea de maximizar o minimizar ha sido fundamental en el cálculo y la física, no estaba formalmente consolidada. Sus primeros usos para resolver problemas prácticos
en la epóca moderna fueron durante la Segunda Guerra Mundial, donde se utilizó para
resolver asuntos de logística y planificación militar, dando origen a la investigación operativa (\textit{Operation Research}, en inglés) \cite{Petropoulos_2023}.
La investigación operativa integraba matemáticas, estadística y experimentación para optimizar operaciones militares, y fue pionera en usar métodos cuantitativos
para la toma de decisiones óptimas en organizaciones. Tomando inspiración en procesos naturales, aleatorios o en el comportamiento humano, surgieron los algoritmos metaheurísticos.
Estos algoritmos dieron lugar a nuevos métodos para resolver problemas de optimización, alcanzando soluciones aproximadas al óptimo en tiempos mucho menores \cite{Tomar2023metaheuristics}.
Este avance significó un gran salto en la capacidad de resolver problemas complejos y reales, que hasta ese momento eran intratables por los métodos exactos.


% Growing interest in optimization in recent years and machine learning influence
Dada la relevancia actual de los problemas de optimización, el interés por este campo se ha desarrollado considerablemente en los últimos años. Un estudio reveló que en los últimos 30 años la cantidad de investigaciones, árticulos y
avances importantes relacionados con este tema se ha incrementado exponencialmente \cite{weinand2021research}. El surgimiento del \textit{machine learning} como herramienta para abordar problemas complejos y de alta dimensionalidad
fue un factor clave en el campo de la optimización. La estrecha relación entre estos dos campos ha sido un factor clave en este desarrollo. Los métodos de aprendizaje han logrado modelar relaciones complejas, expandiendo
significativamente el espectro de problemas de optimización que pueden resolverse \cite{bengio2020machinelearningcombinatorialoptimization}.

% Solving optimization problems nowadays
En la actualidad existen una gran cantidad de herramientas y softwares para resolver problemas de optimización. Cada una enfocada en resolver uno o varios de los tipos de problemas
de optimización existentes (programación lineal, programación entera, programación no lineal, optimización robusta y adaptiva, etc). Estas herramientas son lo suficientemente avanzadas y eficientes para manejar problemas relativamente grandes
y complejos en tiempos razonables. En la actualidad resolver problemas de optimización complejos consta de tres pasos \cite{zhang2024solvinggeneralnaturallanguagedescriptionoptimization}. Primero un
experto en el dominio del problema debe analizar los requerimientos y objetivos prácticos, convirtiendolos en una descripción formal del problema. Después se extrae la información relevante de la descripción y se representa en alguno de los lenguajes
de modelado existentes, como \textit{Python}, \textit{AMPL}, \textit{Julia}, etc. Finalmente se usan las herramientas anterioremente mencionadas para alcanzar la solución final.

% The LLM usage in optimizations problems

Desde hace mucho tiempo ha existido un gran interés en reducir la brecha que hay entre el lenguaje natural y el modelado algebraíco o algorítmico de una computadora.
De hecho, en 1996, Eugene C. Freuder describió la idea de que “el usuario especifique el problema y el ordenador lo resuelva” como el “Santo Grial” de la programación de restricciones \cite{tsouros2023holygrail20natural}.
Hace algunos años se dieron pequeños pasos persiguiendo este objetivo, como sistemas expertos que intentaban asistir en el entendimiento e interacción de un modelo matemático usando lenguaje natural \cite{Dantzig1951}.
Sin embargo el avance en el procesamiento y entendimiento del lenguaje natural ha sido un proceso lento debido a su complejidad. Hasta hace pocos años tareas como traducción de textos, análisis sintáctico de oraciones y el
reconocimiento de entidades habían progresado bastante llegando a obtener buenos resultados. Pero todavía analizar un problema complejo, que tiene contexto, ambiguedades e ideas implícitas parecía algo lejano. A partir del año 2018
aparecieron los modelos de lenguaje preentrenados (\textit{pre-trained language models}, en inglés) como \textit{BERT} y \textit{GPT-2} que lograron un desempeño sobresalientes en tareas de procesamiento y generación de lenguaje. Estos modelos fueron escalando en tamaño
hasta llegar a los \textit{Large Language Models} (LLM). Al aumentar el número de parámetros y el tamaño del corpus de entrenamiento, estos modelos no solo escalaron en sus capacidades para entender, generar y manipular textos, sino que desarrollaron “habilidades” que los modelos
más pequeños no presentaban \cite{zhao2025surveylargelanguagemodels}. Dichas “habilidades” hacen referencia a un aumento en el desempeño al utilizar técnicas de \textit{prompting} como \textit{in-context learning} y \textit{few-shots}.


\newpage

\subsection*{Problema}

A pesar de los avances en técnicas y herramientas desarrollados recientemente en el campo de la optimización, hay un gran
problema que todavía persiste: estas herramientas requieren una modelación formal del problema para poder resolverlo. Sin embargo,
los problemas de la vida real no suelen seguir este formato, con frecuencia, describir con exactitud el objeto de optimización resulta complicado incluso para expertos.
Esta dificultad proviene, en gran medida, de la complejidad intrínseca de muchos casos prácticos. Es habitual que incluyan un elevado número de variables, restricciones
implícitas o interrelaciones complejas.
Por lo tanto, en la práctica, se necesita conocimientos especializados en el campo de la optimización y en el uso de dichas herramientas, para poder entender el problema y llevarlo
a una modelación formal.

Este proceso es generalmente costoso en cuestiones de tiempo y recursos, así como demandar la intervención de profesionales con experiencia. Debido a esto existe una barrera entre empresas, negocios o
instituciones, que no cuentan con recursos o personal experto en este campo, y la posibilidad de acceder a las tecnologías de optimización. Siendo este acceso un aspecto
significativo para facilitar la toma de decisiones, mejorar la eficiencia de sus procesos y solucionar próblematicas complejas.


\subsection*{Motivación}
Con este trabajo de tesis se realiza una propuesta para aumentar la accesibilidad a las tecnologías de optimización. Permitiendo a empresas, negocios, investigadores o cualquier persona,
solucionar problemas prácticos de su contexto de forma sencilla e intuitiva. En general, que usuarios no expertos, puedan resolver problemas de optimización a partir de una conversación.
Se busca que a través de un pequeño intercambio, se pueda extraer toda la información del problema y resolverlo automáticamente, dando en todo momento retroalimentación sobre el proceso. Esto permite
solucionar el problema incluso si el usuario no posee un entendimiento fundamental del mismo.



\subsection*{Antecedentes}
El empleo de los LLMs para permitir que usuarios no especializados aborden problemas reales complejos ha sido objeto de investigaciones previas en la Facultad de Matemáticas y Computación de la Universidad de
La Habana. En este contexto, se han desarrollado asistentes virtuales impulsados por LLMs con el propósito de facilitar la comprensión y el acceso a la información de la legislación cubana, así como de optimizar
la interacción y la extracción de datos de los complejos cuadros tabulares incluidos en los Anuarios Estadísticos de Cuba.


\subsection*{Objetivos}

El objetivo general de este trabajo es diseñar una herramienta con una estructura multiagente basados en LLMs\@. A partir de un problema de optimización descrito en lenguaje natural
la herramienta debe extraer la información relevante del problema: objetivos, restricciones, variables, etc. Usando esa información para modelar formalmente el problema y resolverlo de forma automática.
Integrado lo anterior con un sistema de retroalimentación que le permita a la herramienta rectificar errores, mejorando la correctitud de las soluciones finales. Para lograr el objetivo general se proponen los
siguientes objetivos específicos:

\begin{itemize}
    \item Realizar un análisis sobre el estado del arte enfocado en la solución de problemas de optimización a partir de lenguaje natural.
    \item Analizar las distintas tecnologías, herramientas y lenguajes de modelado utilizados para resolver los problemas de optimización actuales.
    \item El diseño y la implementación de la arquitectura multiagente propuesta.
    \item El diseño de una interfaz visual para la interacción del cliente.
    \item La implementación de un sistema de manejo y prevención de errores.
    \item Evaluar la efectividad de la herramienta propuesta, y su rendimiento comparado a otras soluciones.
\end{itemize}


\subsection*{Estructura de la tesis}
El resto del trabajo de tesis está estructurado de la siguiente manera. En el Capítulo~\ref{chapter:state-of-the-art} se realiza una comparación de los diferentes enfoques utilizados para resolver problemas de optimización
a partir de lenguaje natural. Se analizan las principales técnicas de cada enfoque para cada objetivo específico del problema y se muestran las principales dificultades existentes en este campo. Luego, en el Capítulo~\ref{chapter:proposal}
se muestra la arquitectura propuesta, incluyendo una análisis de las estrategias y técnicas utilizadas. En el Capítulo ~\ref{chapter:implementation} se exponen los resultados alcanzados por el modelo con respecto a otros trabajos similares, destacando
como se afecta el desempeño a partir de las las diferentes técnicas utilizadas. Por último se presentan las conclusiones y recomendaciones futuras.