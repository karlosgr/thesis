\begin{recomendations}
    Uno de las principales limitantes existentes en este trabajo y en otros similares encontrados en el estado del arte, es una pobre evaluación en problemas reales prácticos.
    A pesar de que los conjuntos de problemas escogidos son variados y cubren diferentes tipos de problemas de optimización, estan lejos de representar la complejidad y tamaños que se pueden encontrar en problemas de situaciones reales, con miles de variables y restricciones.
    Por lo tanto, se recomienda realizar evaluaciones en problemas reales de mayor tamaño y complejidad, en la medida de lo posible, con el objetivo de validar el desempeño de la solución en escenarios más desafiantes y representativos.
    También se sugiere explorar distintas ideas para corregir los errores más comunes que se mencionan en este trabajo que cometen los modelos a la hora de lidiar con problemas de optimización.
    Finalmente, se recomienda investigar alternativas para mejorar la eficacia de la solución propuesta, en preparación para su posible evaluación en problemas reales de mayor escala.
\end{recomendations}
