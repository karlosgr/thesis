\begin{conclusions}
    El presente trabajo de tesis abordó el desafío fundamental de resolver problemas de optimización a partir de descripciones en lenguaje natural.
    Se desarolló con éxito una solución que funciona como una interfaz lingüística para dichos problemas, acorde a los objetivos del trabajo.
    Se utilizó como base la investigación realizada sobre la literatura para el diseño de la arquitectura, el formato de modelación formal propuesto y otras técnicas complementarias que se usaron en la solución.
    Se implementó usando una estructura multiagente, integrando modelos de lenguaje con herramientas y \textit{solvers} externos.
    Se agregó además un mecanismo de conversación multi-ronda para asistir la falta de conocimientos del usuario, acorde con los objetivos de este trabajo.
    Se cumplió con el objetivo de evaluar la solución propuesta mediante experimentos usando conjuntos de datos seleccionados del estado del arte.
    Donde se alcanzó resultados superiores en términos de precisión con respecto a trabajos previos.
    Se realizó un estudio ablacionado adicional para analizar el impacto de las técnicas y estrategias utilizados en la solución.
    Todo lo anterior valida el cumplimiento de los objetivos generales y específicos planteados al inicio de este trabajo de tesis.
\end{conclusions}
